%%%%%%%%%%%%%%%%%%%%%%%%%%%%%%%%%%%%%%%%%%%%%%%%%%%%%%%%%%%%%%%%%%%%%%%%%%%%%
% Chapter 1: Introducción 
%%%%%%%%%%%%%%%%%%%%%%%%%%%%%%%%%%%%%%%%%%%%%%%%%%%%%%%%%%%%%%%%%%%%%%%%%%%%%%%

%---------------------------------------------------------------------------------
\section{Herramientas para el apoyo del profesorado de informática}
\label{1:sec:1}

En los últimos años se han desarrollado las nuevas tecnologías, lo que ha permitido que lleguen nuevas herramientas de aprendizaje y de apoyo a la docencia. Es el caso de la exitosa plataforma Moodle, un LCMS, sigla de Learning Content Management System. 

El Moodle, como otros LCMS, se utiliza para crear y manejar el contenido de un programa educativo, como por ejemplo un curso. Normalmente se crean partes de contenido en forma de módulos que se pueden personalizar.

Existen otras herramientas como el LMS sigla de Learning Management System. El LMS es un software instalado en un servidor que se emplea para administrar, distribuir y controlar las actividades de formación de una institución. Un LCMS suele estar integrado en un LMS.

Las principales funciones del LMS son: gestionar usuarios, recursos, materiales y actividades de formación, administrar el acceso, controlar y hacer seguimiento del proceso de aprendizaje, realizar evaluaciones, generar informes, gestionar servicios de comunicación como foros de discusión, videoconferencias y correo electrónico entre otros.

Como ejemplos reales de las definiciones mencionadas, se encuentra el servicio LCMS de Google, Google Classroom que está basado en Moodle, con la diferencia de que usa la Suite de Google Drive para el apoyo de almacenamiento y gestión de tareas.

En el caso del apoyo al profesorado de informática encontramos los VPL sigla de Virtual Programing Lab, \cite{B14} que es un módulo para la plataforma Moodle. Los VPL permiten editar y ejecutar el código fuente de las tareas de programación en el navegador, además permite que el profesor pueda corregir sin necesidad de descargar el código fuente para ejecutarlo, también se permite añadir restricciones como pegar código externo y permite comparar la similitud entre códigos de los alumnos, entre otras cosas.

Por último, la única herramienta para la gestión de tareas está desarrollada por Github, se trata de Github Classroom, una plataforma aprovecha las organizaciones y repositorios de Github como estructura para las aulas y tareas, haciendo uso de la herramienta de control de versiones Git. En el siguiente punto hablaremos en profundidad de Github Education y Classroom.


%---------------------------------------------------------------------------------
\section{Github Education}
\label{1:sec:2}

Github Education \cite{B12} es una comunidad desarrollada por Github  para el mundo de la enseñanza y que contiene varias herramientas  y servicios para apoyar a los profesores en su labor:

\begin{itemize}
  \item GitHub Classroom \cite{B13}
  \item Classroom Desktop
  \item Teachers Pet
  \item Education Comunity \cite{B12}
  \item Student Pack
\end{itemize}

\subsection{Classroom}

Classroom es una herramienta destinada a profesores para gestionar
el uso educativo de GitHub. Simplifica la asignación de tareas,
automatizando la creación de repositorios git usando las organizaciones
de Github como aulas y los repositorios como asignaciones. 

Es una
herramienta útil y sencilla de usar, tanto para profesores como
para alumnos, pero tiene ciertos defectos. 

\begin{itemize}
\item Por ejemplo, no se puede
crear un repositorio de evaluación que contenga las tareas de todos
los alumnos, para eso se debe usar Classroom Desktop, es decir, se
requiere de otra herramienta para una única funcionalidad. 

\item Tampoco
se puede acceder a los enlaces de travis, si la práctica requiere
su uso. 

\item Por último, tiene un sistema para asociar información a
cada alumno que es bastante complejo de usar, ya que requiere
introducir a mano cada usuario y la información que se desea asociar.
\end{itemize}

\subsection{Teachers Pet}
Teachers Pet es un CLI destinado a la administración de Education vía terminal, pero dejaron de desarrollarla porque para realizar ciertas tareas requería demasiadas opciones y los comandos eran demasiado largos.

\subsection{Education community}
Education Comunity es un foro para el intercambio de información entre alumnos, profesores, investigadores y desarrolladores.

\subsection{Student Pack}
Student Pack \cite{B15} es un paquete que ofrece Github a los alumnos para
mejorar su experiencia haciendo uso de Github para el desarrrollo
de tareas.
