%%%%%%%%%%%%%%%%%%%%%%%%%%%%%%%%%%%%%%%%%%%%%%%%%%%%%%%%%%%%%%%%%%%%%%%%%%%%%
% Chapter 5: Summary and Conlusions
%%%%%%%%%%%%%%%%%%%%%%%%%%%%%%%%%%%%%%%%%%%%%%%%%%%%%%%%%%%%%%%%%%%%%%%%%%%%%%%

%++++++++++++++++++++++++++++++++++++++++++++++++++++++++++++++++++++++++++++++
The second version of the ghedsh gem has succesfully achieved the objectives outlined
before: improve its initial architecture to prepare it for changes and enable
incorporating future functionalities. Moreover, this will give support to the evaluation
process through GitHub Education’s methodologies. The improvement will offer
funcionalities that will most certainly be difficult to find in others shell prompts.
This has significant implications and contributes to the educational community.
\bigskip

Feedback is key in the learning process. Thus, an effective and efficient communication
with students can make a real difference. With ghedsh it is easier to access the interface
provided by GitHub, which allows to comment lines and portions of the code in the
respective allocations. Hence, the teacher corrections are qualitatively enhanced. In
this way, students can easliy ask questions about the code, but what is more, to
understand how it works.
\bigskip

Additionally, it is planned that ghedsh will be subject to continued further
development as it has been optimised. However, there is still room for improvement
which highly increases the potential of such tool.
\bigskip

For future work lines, some aspects may be considered:
\begin{itemize}
  \item Guide the design towards Open/Close (OCP) principle. Therefore ghedsh will
  be more extensible (in terms of functionality) including plug-ins developed by
  others.
  \item Provide fully structured usage help within the tool.
  \item Include extra information about students, so their profile in “Campus virtual”
  can be visible.
  \item Improve the tool current test structure.
  \item Add extra functionality to the commands that support the evaluation process, by including a set of libraries
  that perform specific tasks for this process.
\end{itemize}


