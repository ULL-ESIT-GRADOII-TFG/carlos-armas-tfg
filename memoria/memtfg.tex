\documentclass[spanish,a4paper,14pt,oneside]{extreport}

%%%%%%%%%%%%%%%%%%%%%%%%%%%%%%%%%%%%%%%%%%%%%%%%%%%%%%%%%%%%%%%%%%%%%%%%%%%%%%%
\usepackage[dvips]{graphicx}
\usepackage[dvips]{epsfig}
\usepackage[utf8]{inputenc}
\usepackage[spanish]{babel}
\usepackage{alltt}
\usepackage{algorithm}
\usepackage{algorithmic}
\usepackage{multirow}
\usepackage{hyperref}
\usepackage{color}
\usepackage[top=2cm, bottom=2cm, left=2cm, right=2cm]{geometry}

%%%%%%%%%%%%%%%%%%%%%%%%%%%%%%%%%%%%%%%%%%%%%%%%%%%%%%%%%%%%%%%%%%%%%%%%%%%%%%%

\newcommand{\SONY}{{\sc Sony}}
\newcommand{\MICROSOFT}{{\sc Microsoft}}
\newcommand{\GCC}{\textsf{\textsc{G}CC}}
\newcommand{\INTEL}{\textsf{\textsc{I}ntel}}

%%% Traducimos el pseudocodigo
\renewcommand{\algorithmicwhile}{\textbf{mientras}}
\renewcommand{\algorithmicend}{\textbf{fin}}
\renewcommand{\algorithmicdo}{\textbf{hacer}}
\renewcommand{\algorithmicif}{\textbf{si}}
\renewcommand{\algorithmicthen}{\textbf{entonces}}
\renewcommand{\algorithmicrepeat}{\textbf{repetir}}
\renewcommand{\algorithmicuntil}{\textbf{hasta que}}
\renewcommand{\algorithmicelse}{\textbf{en otro caso}}
\renewcommand{\algorithmicfor}{\textbf{para}}

%\newcommand{\RETURN}{\textbf{retornar} }
\newcommand{\RET}{\STATE \textbf{retornar} }
\newcommand{\TO}{\textbf{hasta} }
\newcommand{\AND}{\textbf{y} }
\newcommand{\OR}{\textbf{o} }

%%%%%%%%%%%%%%%%% Creamos un entorno para listar código fuente %%%%%%%%%%%%%%%
\newenvironment{sourcecode}
{\begin{list}{}{\setlength{\leftmargin}{1em}}\item\scriptsize\bfseries}
{\end{list}}

\newenvironment{littlesourcecode}
{\begin{list}{}{\setlength{\leftmargin}{1em}}\item\tiny\bfseries}
{\end{list}}

\newenvironment{summary}
{\par\noindent\begin{center}\textbf{Abstract}\end{center}\begin{itshape}\par\noindent}
{\end{itshape}}

\newenvironment{keywords}
{\begin{list}{}{\setlength{\leftmargin}{1em}}\item[\hskip\labelsep \bfseries Keywords:]}
{\end{list}}

\newenvironment{palabrasClave}
{\begin{list}{}{\setlength{\leftmargin}{1em}}\item[\hskip\labelsep \bfseries Palabras clave:]}
{\end{list}}


%%%%%%%%%%%%%%%%%%%%%%%%%%%%%%%%%%%%%%%%%%%%%%%%%%%%%%%%%%%%%%%%%%%%%%%%%%%%%%%
% Format
%%%%%%%%%%%%%%%%%%%%%%%%%%%%%%%%%%%%%%%%%%%%%%%%%%%%%%%%%%%%%%%%%%%%%%%%%%%%%%%
%\usepackage{showframe}
%\marginparwidth 0mm
%%\topmargin -4 mm
%\topmargin -21 mm
%\headheight 10 mm
%\headsep 10 mm

%\textheight 229 mm
%\textheight 246 mm

%\oddsidemargin -5.4 mm
%\evensidemargin -5.4 mm
%\oddsidemargin 5 mm
%\evensidemargin 5 mm

%\oddsidemargin -3 mm
%\evensidemargin -3 mm

%\textwidth 17 cm
%\textwidth 15 cm
%\columnsep 10 mm

\input{amssym.def}

%%%%%%%%%%%%%%%%%%%%%%%%%%%%%%%%%%%%%%%%%%%%%%%%%%%%%%%%%%%%%%%%%%%%%%%%%%%%%%%

\begin{document}

%%%%%%%%%%%%%%%%%%%%%%%%%%%%%%%%%%%%%%%%%%%%%%%%%%%%%%%%%%%%%%%%%%%%%%%%%%%%%%%
% First Page
%%%%%%%%%%%%%%%%%%%%%%%%%%%%%%%%%%%%%%%%%%%%%%%%%%%%%%%%%%%%%%%%%%%%%%%%%%%%%%%

\pagestyle{empty}
\thispagestyle{empty}


\newcommand{\HRule}{\rule{\linewidth}{1mm}}
\setlength{\parindent}{0mm}
\setlength{\parskip}{2.5mm}

\vspace*{\stretch{0.5}}

\begin{center}
\includegraphics[scale=0.8]{images/logo_vertical}\\[10mm]
{\Huge Trabajo de Fin de Grado}
\end{center}

\HRule
\begin{flushright}
  {\Huge GitHub Education Shell: ghedsh} \\[2.5mm]
  {\Large Carlos de Armas Hernández} \\[5mm]


\end{flushright}
\HRule
\vspace*{\stretch{2}}
\begin{center}
  \Large La Laguna, \today
\end{center}

\setlength{\parindent}{5mm}

%%%%%%%%%%%%%%%%%%%%%%%%%%%%%%%%%%%%%%%%%%%%%%%%%%%%%%%%%%%%%%%%%%%%%%%%%%%%%%%
% Signature page (add the official stamp)
%%%%%%%%%%%%%%%%%%%%%%%%%%%%%%%%%%%%%%%%%%%%%%%%%%%%%%%%%%%%%%%%%%%%%%%%%%%%%%%
\newpage
%\cleardoublepage
\thispagestyle{empty}

D. {\bf Casiano Rodríguez León}, con N.I.F. 42.020.072-S profesor Catedrático de Universidad adscrito al Departamento de Lenguajes y Sistemas Informáticos de la Universidad de La Laguna, como tutor

\bigskip
\bigskip
{\bf C E R T I F I C A}

\bigskip
\bigskip
\bigskip
Que la presente memoria titulada:

\bigskip
``{\it GitHub Education Shell: ghedsh.}''

\bigskip
\bigskip
\bigskip

\noindent ha sido realizada bajo su dirección por D. {\bf Carlos de Armas Hernández},
con N.I.F. 54.062-352-R.

\bigskip
\bigskip

Y para que así conste, en cumplimiento de la legislación vigente y a los efectos oportunos firman la presente en La Laguna a \today.

%\cleardoublepage
\newpage
%%%%%%%%%%%%%%%%%%%%%%%%%%%%%%%%%%%%%%%%%%%%%%%%%%%%%%%%%%%%%%%%%%%%%%%%%%%%%%%
\thispagestyle{empty}

{ \flushright

\begin{LARGE}
Agradecimientos
\end{LARGE}

\hspace{3mm}

\begin{large}

\hspace{3mm}
XXXXXXXXXXXXXXXXXXXXXXXXXXXXX

\hspace{3mm}
XXXXXXXXXXXXXXXXXXXXXXXXXXXXX
\end{large}

}

%%%%%%%%%%%%%%%%%%%%%%%%%%%%%%%%%%%%%%%%%%%%%%%%%%%%%%%%%%%%%%%%%%%%%%%%%%%%%%%%%
\newpage

\begin{huge}
Licencia
\end{huge}

\begin{center}
\includegraphics[scale=1.5]{images/by-nc_88x31}\\[10mm]
{\Large \copyright~Esta obra está bajo una licencia de Creative Commons Reconocimiento-NoComercial 4.0 Internacional.
}
\end{center}

%%%%%%%%%%%%%%%%%%%%%%%%%%%%%%%%%%%%%%%%%%%%%%%%%%%%%%%%%%%%%%%%%%%%%%%%%%%%%%%
\newpage  %\cleardoublepage
\begin{abstract}
{\em

Este Trabajo de Fin de Grado tiene como objetivo mejorar la versión previa de la gema «ghedsh». Una
interfaz de línea de comandos (en inglés, command-line interface, CLI) desarrollada para soportar las metodologías de GitHub Education y que facilita la asignación de tareas, así como la revisión cualitativa y cuantitativa de las mismas.

Para ello, en una primera etapa, se ha llevado a cabo una refactorización del código fuente. De esta manera, otros desarrolladores podrán sumarse al proyecto para crear código limpio y mantenible.

En cuanto a la segunda etapa, ésta ha consistido en añadir funcionalidades a la gema, dando prioridad a aquellas que ofrecen solución a las limitaciones que poseen otras herramientas similares. 
}

\begin{palabrasClave}
Git, GitHub, Ruby, CLI, GitHub Education, Profesores, Evaluación.
\end{palabrasClave}

\end{abstract}
%%%%%%%%%%%%%%%%%%%%%%%%%%%%%%%%%%%%%%%%%%%%%%%%%%%%%%%%%%%%%%%%%%%%%%%%%%%%%%%

%%%%%%%%%%%%%%%%%%%%%%%%%%%%%%%%%%%%%%%%%%%%%%%%%%%%%%%%%%%%%%%%%%%%%%%%%%%%%%%
\newpage  %\cleardoublepage
\begin{summary}
{\em

This End-of-Degree Project aims to improve the previous version of "ghedsh" gem, an easy to use command-line interface (CLI) that supports GitHub Education's methodologies, facilitating the creation of assignments and also their qualitative and quantitative review.

To make this happen, in a first stage, a refactoring of existing source code has been carried out. In this way, other developers can join the project to create clean and maintainable code.

Regarding the second stage, the main task was adding functionalities, giving priority to those that solve limitations of other similar tools.
}

\begin{keywords}
Git, GitHub, Ruby, CLI, GitHub Education, Teachers, Grading.
\end{keywords}

\end{summary}
%%%%%%%%%%%%%%%%%%%%%%%%%%%%%%%%%%%%%%%%%%%%%%%%%%%%%%%%%%%%%%%%%%%%%%%%%%%%%%%

%%%%%%%%%%%%%%%%%%%%%%%%%%%%%%%%%%%%%%%%%%%%%%%%%%%%%%%%%%%%%%%%%%%%%%%%%%%%%%%
\newpage{\pagestyle{empty}}
\thispagestyle{empty}

%%%%%%%%%%%%%%%%%%%%%%%%%%%%%%%%%%%%%%%%%%%%%%%%%%%%%%%%%%%%%%%%%%%%%%%%%%%%%%%


\pagestyle{myheadings} %my head defined by markboth or markright
% No funciona bien \markboth sin "twoside" en \documentclass, pero al
% ponerlo se dan un montón de errores de underfull \vbox, con lo que no se
% ha puesto.
\markboth{Carlos de Armas Hernández}{GitHub Education Shell: ghedsh.}

%%%%%%%%%%%%%%%%%%%%%%%%%%%%%%%%%%%%%%%%%%%%%%%%%%%%%%%%%%%%%%%%%%%%%%%%%%%%%%%
%Numeracion en romanos
\renewcommand{\thepage}{\roman{page}}
\setcounter{page}{1}

%%%%%%%%%%%%%%%%%%%%%%%%%%%%%%%%%%%%%%%%%%%%%%%%%%%%%%%%%%%%%%%%%%%%%%%%%%%%%%%

\tableofcontents

%%%%%%%%%%%%%%%%%%%%%%%%%%%%%%%%%%%%%%%%%%%%%%%%%%%%%%%%%%%%%%%%%%%%%%%%%%%%%%%
\newpage{\pagestyle{empty}}

\listoffigures

%%%%%%%%%%%%%%%%%%%%%%%%%%%%%%%%%%%%%%%%%%%%%%%%%%%%%%%%%%%%%%%%%%%%%%%%%%%%%%%
\newpage{\pagestyle{empty}}

\listoftables

%%%%%%%%%%%%%%%%%%%%%%%%%%%%%%%%%%%%%%%%%%%%%%%%%%%%%%%%%%%%%%%%%%%%%%%%%%%%%%%
\newpage{\pagestyle{empty}}

%%%%%%%%%%%%%%%%%%%%%%%%%%%%%%%%%%%%%%%%%%%%%%%%%%%%%%%%%%%%%%%%%%%%%%%%%%%%%%%
%Numeracion a partir del capitulo I
\renewcommand{\thepage}{\arabic{page}}
\setcounter{page}{1}


\chapter{Introducción}
\label{chapter:intro}

%%%%%%%%%%%%%%%%%%%%%%%%%%%%%%%%%%%%%%%%%%%%%%%%%%%%%%%%%%%%%%%%%%%%%%%%%%%%%
% Chapter 1: Introducción 
%%%%%%%%%%%%%%%%%%%%%%%%%%%%%%%%%%%%%%%%%%%%%%%%%%%%%%%%%%%%%%%%%%%%%%%%%%%%%%%

%---------------------------------------------------------------------------------
\section{Antecedentes}
\label{1:sec:1}

Git es un software de control de versiones de código abierto ampliamente utilizado, diseñado por Linus Torvalds.
El propósito del control de versiones es llevar un registro de los diversos cambios que se realizan sobre
los elementos de un proyecto y, de esta manera, facilitar la coordinación del trabajo que varias personas
realizan sobre el mismo.
\bigskip

Por otro lado, GitHub es una plataforma de desarrollo colaborativo empleada para alojar proyectos que utilizan el control de verisones Git.
Esta plataforma ofrece numerosos servicios que dan soporte al trabajo colaborativo. Entre los más destacados tenemos: repositorios, documentación,
gestión de incidencias, gestión de equipos, chats, notificaciones, alojamiento de páginas web y tableros para organizar de manera flexible y visual las tareas que componen un proyecto.
\bigskip

Asimismo, GitHub dispone de un programa denominado {\it GitHub Education} que ofrece una variedad de herramientas tanto para el profesorado como para estudiantes.
\bigskip

A los docentes se les ofrece la posibilidad de crear organizaciones para cada clase, mediante repositorios privados que se crean automáticamente para el conjunto de estudiantes y equipos,
a partir de las asignaciones realizadas por los profesores de las asignaturas a través la herramienta {\it GitHub Classroom }.
\bigskip

En cuanto a los estudiantes, se les da acceso gratuito a herramientas de desarrollo, plataformas de alojamiento y nuevas características en GitHub gracias a {\it Student Developer Pack }.
\bigskip

En este contexto, multitud de docentes se plantean la necesidad del desarrollo de metodologías e implantación de herramientas
que asistan al profesorado en la evaluación de las tareas y actividades de programación, principalmente.
%---------------------------------------------------------------------------------
\section{Estado actual del arte}
\label{1:sec:2}
Con el fin de integrar el uso del control de versiones en las aulas, GitHub desarrolló una herramienta de línea de comandos bajo el nombre de {\it Teachers Pet}. Ésta herramienta proponía que cada aula fuese una
organización, lo que permitía a los profesores (que serían los propietarios de la organización) administrar todos los repositorios dentro de la misma. Con esta forma de trabajar se lograban dos objetivos:
\begin{itemize}
  \item Facilitar a los profesores la distribución de código de inicio a los alumnos.
  \item Los profesores eran capaces de acceder a las tareas de los alumnos para resolver preguntas y comprobar el progreso.
\end{itemize}
Sin embargo, cayó en desuso y se dejó de desarrollar. La principal razón fue que los comandos se hacían excesivamente largos en determinadas tareas, puesto que era necesario especificar múltiples opciones.
\bigskip

Por otro lado, también existen diversas alternativas desarrolladas por la comunidad. Éstas son:
\begin{itemize}
  \item {\it ghi} \cite{B1} (GitHub Issues): proporciona una manera fácil de gestionar las incidencias ({\it issues}) de los repositorios desde la terminal del usuario, utilizando su editor de preferencia. Es posible llevar a cabo diversas acciones como, por ejemplo,
  listar incidencias, abrirlas y cerrarlas y comentar en ellas, entre otras.
  \item {\it ghs} \cite{B2} (GitHub Search): es una utilidad que permite realizar búsquedas de repositorios alojados en GitHub. Admite diversas opciones para limitar las búsquedas en el ámbito deseado, como organizaciones y usuarios, por ejemplo.
\end{itemize}

Actualmente, GitHub dispone de una plataforma, {\it GitHub Classroom} \cite{B3}, que mejora la experiencia de usuario respecto a las herramientas anteriormente nombradas, dado que dispone de una interfaz amigable en el navegador que simplifica
la configuración de las aulas y asignaciones. No obstante, en esta plataforma existen algunas limitaciones:
\begin{itemize}
  \item No dispone de alguna funcionalidad dentro de ella que permita al profesor crear un repositorio de evaluación para calificar las tareas.
  \item No da soporte a herramientas de integración continua, como {\it Travis CI}, {\it CircleCI}, {\it Jenkins}, etc.
  \item En muchos casos, el nombre de usuario que ha escogido el alumno al registrarse en GitHub no permite identificarlo. Además, el sistema para añadir información adicional del alumno es incómodo de usar,
  puesto que la información se inserta individualmente.
\end{itemize}
%---------------------------------------------------------------------------------
\section{Objetivos y actividades a realizar}
\label{1:sec:3}

El principal objetivo de este Trabajo de Fin de Grado es desarrollar una interfaz de línea de comandos, bajo el nombre de {\it GitHub Education Shell},
que utilice la misma metodología que {\it Teachers Pet} y {\it GitHub Classroom} y que aporte una solución a las limitaciones que poseen dichas herramientas, así como priorizar
la realización de tareas a gran escala más comunes en entornos educativos.
\bigskip

Para lograr dichos objetivos, se han definido diferentes actividades:
\begin{itemize}
  \item {\bf A1.} Estudio del funcionamiento de la API ({\it Application Programming Interface}) de GitHub y su implementación oficial en Ruby, {\it Octokit}.
  \item {\bf A2.} Analizar aplicaciones similares e identificar aspectos a mejorar.
  \item {\bf A3.} Comprensión del código de la última versión de la gema {\it ghedsh}.
  \item {\bf A4.} Estudio de aspectos positivos y negativos del diseño de dicha gema.
  \item {\bf A5.} Refactorización del código fuente, aplicando patrones de diseño.
  \item {\bf A6.} Estudio de nuevas funcionalidades para incorporar.
  \item {\bf A7.} Implementación de las funcionalidades escogidas.
  \item {\bf A8.} Definir estructura básica de pruebas.
  \item {\bf A9.} Documentación del código.
\end{itemize}
\bigskip

En la siguiente tabla se muestra el plan de trabajo con la duración de las actividades:
%--------------------------------------------------------------------------
\begin{table}[!ht]
\begin{center}
\begin{tabular}{|p{25mm}|p{50mm}|} \hline 
\textbf{Objetivo} & \textbf{Fecha} \\ \hline

A1, A2, A3 & Febrero \\
\hline

A4, A5 & Marzo, Abril 
\\
\hline

A6, A7 & Mayo
\\
\hline

A8, A9 & Junio
\\
\hline

\end{tabular}
\end{center}
\caption{Tabla resumen del plan de trabajo}
\label{table:resOthers}
\end{table}

%---------------------------------------------------------------------------------
\section{Tecnología empleada}
\label{1:sec:4}
 
En cuanto a la tecnología empleada, el lenguaje de programación escogido para el desarrollo ha sido {\it Ruby},
puesto que la versión anterior también está escrita en este lenguaje.
\bigskip

Para que el programa utilice los datos de GitHub del usuario, se ha utilizado la API REST v3. En concreto,
la librería oficial escrita en Ruby, \cite{B4}, la cual proporciona una gran cantidad de métodos para realizar diferentes acciones, como, por ejemplo,
crear repositorios y administrar su configuración, crear equipos y acceder a la información del usuario, entre otros.
Para la autenticación del usuario, se generará un {\it token} automáticamente con los permisos requeridos para el uso del programa.

Por otro lado, en cuanto al ecosistema de {\it Ruby}, tenemos:
\begin{itemize}
  \item {\it Rubygems} \cite{B5}: servicio de alojamiento de gemas para la comunidad de Ruby.
  \item {\it Bundler} \cite{B6}: proporciona un entorno para manejar las dependencias de un proyecto Ruby, añadiendo las versiones que son necesarias.
  \item {\it YARD} \cite{B7}: herramienta para la documentción del código fuente.
  \item {\it RSpec} \cite{B8}: librería para la definición de pruebas.
\end{itemize}


%%%%%%%%%%%%%%%%%%%%%%%%%%%%%%%%%%%%%%%%%%%%%%%%%%%%%%%%%%%%%%%%%%%%%%%%%%%%%%%

\chapter{Plan de trabajo}
\label{chapter:dos}

%%%%%%%%%%%%%%%%%%%%%%%%%%%%%%%%%%%%%%%%%%%%%%%%%%%%%%%%%%%%%%%%%%%%%%%%%%%%%%%
% Chapter 2: Desarrollo
%%%%%%%%%%%%%%%%%%%%%%%%%%%%%%%%%%%%%%%%%%%%%%%%%%%%%%%%%%%%%%%%%%%%%%%%%%%%%%%

%++++++++++++++++++++++++++++++++++++++++++++++++++++++++++++++++++++++++++++++
En este capítulo dos se va a describir el desarrollo del proyecto. Se dividirá en dos fases bien diferenciadas, una primera fase que consiste en el análisis y refactorización del código fuente de la versión anterior de
{\it GitHub Education Shell}, y una segunda fase que trata de la incorporación de nuevas funcionalidades a la herramienta.
\bigskip

Por otro lado, también cabe nombrar la metodología de trabajo empleada. Situándonos en el marco de las metodologías ágiles, {\it Scrum} fue la metodología que mejor encajaba, teniendo en cuenta las características del proyecto.
Se ha optado por un desarrollo incremental, en lugar de una planificación y ejecución estricta de las tareas. Además, en numerosas ocasiones,
se produjeron solapamientos de las diferentes partes del desarrollo, en vez de un ciclo secuencial. También fueron frecuentes las reuniones con el tutor, en las que se comentaban tanto avances como dificultades.

%++++++++++++++++++++++++++++++++++++++++++++++++++++++++++++++++++++++++++++++

\section{Primera fase: análisis y refactorización.}
\label{2:sec:1}

En esta sección, se explicará detalladamente el proceso fundamental de la primera fase de este Trabajo de Fin de Grado.


%---------------------------------------------------------------------------------
\subsection{Análisis}
\label{subsec:2.1.1}

El análisis del código fuente correspondiente a la primera versión de {\it ghedsh}, se ha llevado a cabo con la finalidad de identificar aquellas partes mejorables del diseño e implementación,
puesto que una de las prioridades es facilitar el desarrollo colaborativo y, para ello, se requiere que el código sea limpio y fácil de entender, lo más auto-explicativo posible.

\begin{figure}[H]
\begin{center}
%\includegraphics[width=0.47\textwidth]{images/}
\caption{Ramas del repositorio}
%\label{fig:github2}
\end{center}
\end{figure}


La documentación adicional para llevar a cabo los desarrollos de cada iteración, así como los problemas detectados, se anotaban en el apartado de \verb1issues1 con el fin de que quedara constancia de ello y se reflejara el estado en el que se encontraba cada uno.

\begin{figure}[H]
\begin{center}
%\includegraphics[width=1\textwidth]{images/}
\caption{Apartado de issues}
%\label{fig:github3}
\end{center}
\end{figure}

%---------------------------------------------------------------------------------
\subsection{Travis-CI}
\label{subsec:2.1.2}

Como herramienta de integración continua, se ha utilizado Travis-CI, con el fin de asegurarnos el despliegue de la aplicación era satisfactorio tras cada cambio subido a la herramienta de control de versiones (GitHub).

%---------------------------------------------------------------------------------
\subsection{Experiencia de usuario}
\label{subsec:2.1.3}

Por otra parte, el tutor del Trabajo de Fin de Máster ha hecho pruebas reales con el resultado de cada iteración, actuando como {\it Product Owner}. De este modo, se comprobaba el funcionamiento de la aplicación en un entorno real y se recibía un valioso feedback para corregir problemas o hacer mejoras en las siguientes iteraciones.


\newpage
\input{tables/plan.tex}

%%%%%%%%%%%%%%%%%%%%%%%%%%%%%%%%%%%%%%%%%%%%%%%%%%%%%%%%%%%%%%%%%%%%%%%%%%%%%%%

\newpage{\pagestyle{empty}}
\thispagestyle{empty}

\chapter{Desarrollo del proyecto}
\label{chapter:tres}

%%%%%%%%%%%%%%%%%%%%%%%%%%%%%%%%%%%%%%%%%%%%%%%%%%%%%%%%%%%%%%%%%%%%%%%%%%%%%%%
% Chapter 3: Resultados
%%%%%%%%%%%%%%%%%%%%%%%%%%%%%%%%%%%%%%%%%%%%%%%%%%%%%%%%%%%%%%%%%%%%%%%%%%%%%%%

%++++++++++++++++++++++++++++++++++++++++++++++++++++++++++++++++++++++++++++++
Este capítulo se centrará en explicar las características que incorpora {\it ghedsh} tras la etapa de desarrollo tratada en el capítulo anterior.
\bigskip

Se hará una distinción entre comandos del núcleo y comandos incorporados ({\it built-in commands}). Los comandos del núcleo, son aquellos que no trabajan con los datos de GitHub del usuario pero que, sin embargo, son
esenciales desde el punto de vista de la usabilidad y experiencia de usuario con el CLI.
\bigskip

Además, los comandos incorporados sí trabajan con los datos de GitHub del usuario identificado. Permiten realizar diversas tareas, priorizando la rapidez en la ejecución de las mismas y la facilidad de uso de la herramienta.

%---------------------------------------------------------------------------------
\section{Autenticación con credenciales de GitHub}
\label{3:sec:1}

El contenido de esta sección pretende explicar el proceso de autenticación que debe seguir el usuario al usar {\it ghedsh} por primera vez.
\bigskip

Dicho proceso es necesario, puesto que se trabajan con los datos que dispone el usuario en GitHub. Además, la API REST v3 requiere, para ciertas consultas (en especial, modificaciones como crear repositorios, equipos y administrar la configuración), verificar la identidad del usuario. Si no fuera así, se podrían llevar a cabo comportamientos indeseados.
\bigskip

En {\it ghedsh}, se realiza la autenticación con {\it OAuth access token}\cite{B16}, que consiste, en una definición muy simplificada, en una cadena de caracteres alfanuméricos que actúa como una contraseña. No obstante,
en este caso de uso es mucho más potente y segura. Las principales ventajas son:
\begin{itemize}
	\item Es revocable, es decir, el {\it token} puede dejar de ser válido, eliminando el acceso para ese {\it token} en particular, sin que el usuario tenga que cambiar su contraseña en todos sus accesos.
	\item Sus permisos son configurables, esto es, un {\it token} puede ser válido sólo para ciertos recursos de una API. De esta manera, se conceden permisos de forma más controlada.
\end{itemize}

Para sintetizar este apartado, el usuario que utilice por primera vez {\it ghedsh}, debe verificar su identidad mediante sus credenciales (nombre de usuario y contraseña) de GitHub y se
generará de forma automática un {\it token} de acceso con los permisos necesarios para usar la herramienta.

\begin{figure}[H]
	\begin{center}
	\includegraphics[width=0.68\textwidth]{images/login-example}
	\caption{Ejemplo de autenticación al usar ghedsh por primera vez.}
	\label{fig:masterv1}
	\end{center}
\end{figure}

%------------------------------------------------------------------------------------------------------------
\section{Comandos del núcleo de ghedsh}
\label{3:sec:2}

Como se ha indicado en la introducción de este tercer capítulo, se han separado, por un lado, los comandos del núcleo de {\it ghedsh} y, por otro, los comandos característicos de {\it ghedsh}.
\bigskip

En esta sección, se explicará este primer grupo de comandos, encargado de tareas relacionadas con el sistema operativo y, lo más importante, hacer que la herramienta sea agradable de manejar para el usuario. Además, se revisarán aspectos importantes de su implementación.

\subsection{Change directory: cd}
\label{3.2.1}
Análogamente al comando {\it cd} de la {\it Bash}\cite{B17}, que permite cambiar nuestro directorio actual de trabajo, en {\it ghedsh} también existe este comando. No obstante, aunque la idea es similar, existen diferencias a la hora de usarlo.
\bigskip

En nuestro sistema operativo (tipo Unix)\cite{B18}, cuando realizamos la operación {\it cd}, sólo podemos movernos entre directorios (dependiendo de los permisos). Dado que en {\it ghedsh} no existen directorios como tal, hablaremos de contextos.
Los contextos en esta herramienta hacen referencia a nivel de usuario, nivel de organización, nivel de repositorio, etcétera.
\bigskip

Imaginemos por un momento que nuestro usuario se llama \verb ejemplo , disponemos de un repositorio que se llama \verb ejemplo  y una organización denominada \verb ejemplo . Ésto es totalmente válido, puesto que lo que no permite GitHub es que dos usuarios se llamen igual, que el usuario tenga dos repositorios con el mismo nombre o dos organizaciones bajo el mismo nombre.
Entonces, debemos proporcionar alguna manera de desambiguar a qué contexto nos queremos cambiar.
\bigskip

En {\it ghedsh} se ha optado por el siguiente planteamiento: para realizar la operación de
{\it cd}, es necesario especificar el tipo de contexto (nivel) al que queremos cambiarnos, así, aunque el usuario se encuentre en el caso anteriormente comentado, {\it ghedsh} es capaz de saber a qué contexto debe cambiar.
La sintaxis del comando sería \verb cd  \verb <tipo>  \verb <nombre|Regexp> , donde nombre es la cadena de texto que identifica al tipo, o bien, una expresión regular que mostraría las cadenas que han casado.
\bigskip

Los tipos de contexto (pueden ser ampliables) que actualmente se soportan en {\it ghedsh} son:
\begin{itemize}
	\item \textbf{Nivel de usuario}: estando a nivel de usuario, éste se puede cambiar a cualquiera de sus repositorios o a cualquier organización a la que pertenezca, como vemos a continuación:
		\begin{itemize}
			\item Repositorio del usuario: \verb cd   \verb repo  \verb <nombre|/Regexp/> .
			\item Organización del usuario: \verb cd   \verb org  \verb <nombre|/Regexp/> .
		\end{itemize}
	\item \textbf{Nivel de organización}: estando a nivel de una organización de la que es miembro el usuario autenticado, ({\it ghedsh} sabrá que se refiere al entorno de la organización a la que se ha cambiado) se puede mover a:
		\begin{itemize}
			\item Repositorio de la organización: \verb cd   \verb repo  \verb <nombre|/Regexp/> .
			\item Equipo de la organización: \verb cd   \verb team  \verb <nombre|/Regexp/> .
		\end{itemize}
\end{itemize}

Además, si deseamos volver al contexto anterior, haremos de la misma manera que en sistemas operativos tipo Unix: \verb cd  \verb ..  . Hay que tener en cuenta que, actualmete, no se puede realizar la operación de volver al contexto anterior y cambiar a otro de manera simultánea (como en Unix \verb cd  \verb ../another/dir  ), es necesario hacerlo por separado.


\subsubsection{Detalles de implementación}
Puesto que se trata de uno de los comandos más importantes de {\it ghedsh}, se comentarán los aspectos destacados de la implementación del mismo, incluyendo las dificultades encontradas.
\bigskip

Internamente, el comando {\it cd} contiene una pila (stack\cite{B19}) en la que se almacenan todos los contextos previos. La estructura de un contexto se muestra en el siguiente fragmento de código:

\begin{lstlisting}[language=Ruby]
  config = {
    'User' => client.login.to_s,
  	'user_url' => client.web_endpoint.to_s << client.login.to_s,
    'Org' => nil,
    'org_url' => nil,
    'Repo' => nil,
    'repo_url' => nil,
    'Team' => nil,
    'team_url' => nil
  }
\end{lstlisting}

\begin{itemize}
	\item \verb User : permite saber el nombre del usuario autenticado en {\it ghedsh}.
	\item \verb user_url : contiene la URL ({\it Uniform Resource Locator}\cite{B20}) del perfil del usuairo en GitHub.
	\item \verb Org : indica el nombre de la organización actual, si el usuario no está posicionado sobre alguna, el valor es nulo.
	\item \verb org_url : URL de la organización en GitHub.
	\item \verb Repo : nombre del repositorio actual, en caso de estar dentro de alguno.
	\item \verb repo_url : URL del repositorio en GitHub.
	\item \verb Team : nombre del equipo actual si el usuario está posicionado dentro de alguno.
	\item \verb team_url : URL del equipo en GitHub.
\end{itemize}

En esencia, {\it change directory} irá variando estos parámetros para conocer a qué nivel se encuentra el usuario (\verb User  siempre tendrá un valor asignado porque representa el usuario autenticado).
Por ejemplo, para referirnos a un repositorio de una organización en la que el usuario es miembro, tendríamos:
\begin{lstlisting}[language=Ruby]
	config = {
    'User' => client.login.to_s,
  	'user_url' => client.web_endpoint.to_s << client.login.to_s,
    'Org' => "EXAMPLE-ORG",
    'org_url' => nil,
    'Repo' => "repository-within-example-org",
    'repo_url' => nil,
    'Team' => nil,
    'team_url' => nil
  }
\end{lstlisting}
En el caso de un repositorio de usuario, \verb User  tendría valor asignado y \verb Repo  también tendría valor asignado. A diferencia con el caso anterior, \verb Org  sería nulo ya que nos referimos a un repositorio a nivel de usuario.
\bigskip

Una de las dificultades en la implementación de este comando, fue que, antes de reasignar la estructura de datos que respresenta los contextos, era necesario almacenar el contexto actual para poder volver a éste más tarde.
\bigskip

Ruby proporciona dos métodos para copiar/clonar objetos: \verb dup \cite{B21} y \verb clone \cite{B22}. No obstante, realizan una copia superficial del objeto, es decir, crearán un nuevo identificador de objeto pero
el contenido del mismo referenciará al de la entidad original.
\bigskip

Para solucionarlo, se utilizó el módulo {\it Marshal}\cite{B23} de Ruby, que sí realiza una copia profunda del objeto.

\subsection{bash (comando ghedsh)}
\label{3.2.1}

Permite interpretar un comando en la terminal del sistema operativo, sin salir de {\it ghedsh}.
\bigskip

\textbf{Sintaxis:} \verb bash  \verb <comando_terminal>  .
En la figura \ref{fig:bash-example}, se muestra un ejemplo de uso.
\begin{figure}[H]
	\begin{center}
	\includegraphics[width=0.30\textwidth]{images/bash-example}
	\caption{Ejemplo de uso del comando bash.}
	\label{fig:bash-example}
	\end{center}
\end{figure}

%------------------------------------------------------------------------------------------------------------
\section{Comandos incorporados en ghedsh}
\label{3:sec:3}

A lo largo de esta sección, se explicarán individualmente los comandos característicos de {\it ghedsh}. Los comandos característicos o incorporados por {\it ghedsh}
son aquellos que trabajan con los datos de GitHub del usuario de la herramienta. Colaboran estrechamente con la GitHub API REST v3 (véase, para más detalle, la documentación oficial de {\it Octokit}\cite{B24}).
Se explicará la sintaxis para cada uno de ellos y se proporcionará ejemplos de uso.
\bigskip

Como convenio, los parámetros con el formato \verb <parameter> , son obligatorios. Los que tengan el formato \verb [parameter] , son opcionales.
\bigskip

Por otro lado, las expresiones regulares admiten las opciones establecidas por Ruby.

\subsection{clear}
\label{3.3.1}

Realiza la misma tarea que el comando \verb clear  en Bash. Borra la pantalla si es posible y sitúa el cursor en la parte superior izquierda de la pantalla (ignora cualquier parámetro adicional).

\textbf{Sintaxis}: \verb clear  .

\subsection{exit}
\label{3.3.2}

Temina la ejecución de {\it ghedsh}, guardando el contexto actual. Es decir, si el usuario se encontraba dentro de una organización, la próxima vez que entre en {\it ghedsh} estará dentro de la organización.

\textbf{Sintaxis}: \verb exit  .

\subsection{repos}
\label{3.3.3}

Muestra los repositorios según el contexto. Si no se le especifica nungún parámetro, mostrará todos los repositorios. Si se le proporciona una expresión regular, mostrará los nombres de los repositorios que hayan casado.

\textbf{Sintaxis}: \verb repos \verb [/Regexp/] .
\begin{itemize}
	\item Contexto \textbf{organización}: muestra los repositorios de la organización en la que se encuentra el usuario de {\it ghedsh}.
	\begin{figure}[H]
		\begin{center}
		\includegraphics[width=0.60\textwidth]{images/user-repos}
		\caption{Comando repos a nivel de usuario.}
		\label{fig:user-repos}
		\end{center}
	\end{figure}
	\item Contexto \textbf{usuario}: muestra los repositorios del usuario autenticado en {\it ghedsh}.
	\begin{figure}[H]
		\begin{center}
		\includegraphics[width=0.70\textwidth]{images/org-repos}
		\caption{Comando repos a nivel de organización.}
		\label{fig:org-repos}
		\end{center}
	\end{figure}
\end{itemize}

\subsection{new\_repo}
\label{3.3.4}

Crea un nuevo repositorio. La herramienta muestra al usuario dos opciones para crearlo (menú):
\begin{itemize}
	\item {\it Default}: crea un repositorio público.
	\item {\it Custom}: crea un repositorio público o privado, al que es posible añadirle opciones específicas mediante una guía que se le mostrará.
	Si no se quiere especificar alguna de las opciones que se muestran, puede pulsar la tecla retorno y omitir el paso.
\end{itemize}
\textbf{Sintaxis}: \verb new_repo  \verb <nombre_repositorio> .
Disponible para:
\begin{itemize}
	\item Contexto \textbf{usuario}: crea un repositorio para el usuario autenticado.
	\item Contexto \textbf{organización}: crea un repositorio dentro de la organización en la que se encuentre posicionado el usuario.
\end{itemize}
En la figura \ref{fig:create-repo}, vemos el menú de creación del repositorio.
\begin{figure}[H]
	\begin{center}
	\includegraphics[width=0.70\textwidth]{images/create-repo}
	\caption{Menú para la creación de un repositorio.}
	\label{fig:create-repo}
	\end{center}
\end{figure}

\begin{figure}[H]
	\begin{center}
	\includegraphics[width=1\textwidth]{images/custom-repo}
	\caption{Creación de un repositorio de usuario con opciones específicas.}
	\label{fig:custom-repo}
	\end{center}
\end{figure}


\subsection{rm\_repo}
\label{3.3.5}

Elimina el repositorio especificado. Se puede realizar tanto a nivel de \textbf{usuario} como a nivel de \textbf{organización}.

\textbf{Sintaxis}: \verb rm_repo  \verb <nombre_repositorio>  .

\begin{figure}[H]
	\begin{center}
	\includegraphics[width=0.5\textwidth]{images/delete-repo}
	\caption{Eliminación de un repositorio.}
	\label{fig:custom-repo}
	\end{center}
\end{figure}

\subsection{clone}
\label{3.3.6}

Clona repositorios y, si ya existe en el directorio local, realizará \verb git  \verb pull  \verb --all . Recibe como parámetro obligatorio el nombre del repositorio que se desea clonar o una expresión regular, que permitirá clonar todos los repositorios que casen con ella.
Opcionalmente, es posible especificar un directorio dentro de \verb $HOME   de la máquina local. En caso de no exisir, se creará. Por defecto, se clonarán en el directorio actual de la máquina local.

\textbf{Sintaxis}: \verb clone  \verb <nombre_repo|/Regexp/>  \verb [ruta_en_home]  .

Se puede ejecutar en:
\begin{itemize}
	\item Contexto de \textbf{usuario}: clonará los repositorios del usuario autenticado en {\it ghedsh}.
	\item Contexto de \textbf{organización}: clonará los repositorios de la organización en la que se encuentre posicionado.
\end{itemize}

En la figura \ref{fig:clone-example} y \ref{fig:clone-example-org} se muestran ejemplos de uso.

\begin{figure}[H]
	\begin{center}
	\includegraphics[width=0.60\textwidth]{images/clone-example}
	\caption{Ejemplo de clonar un repositorio de usuario.}
	\label{fig:clone-example}
	\end{center}
\end{figure}

\begin{figure}[H]
	\begin{center}
	\includegraphics[width=0.60\textwidth]{images/clone-example-org}
	\caption{Ejemplo de clonar un repositorio de organización.}
	\label{fig:clone-example-org}
	\end{center}
\end{figure}

\subsection{new\_issue}
\label{3.3.7}

Abre el navegador por defecto, situando al usuario en el formulario de creación de un issue. Para ejecutarlo, es necesario estar posicionado sobre un repositorio.

\textbf{Sintaxis}: \verb new_issue  .

Está disponible para:
\begin{itemize}
	\item Contexto \textbf{usuario}: abre la página de {\it issues} del repositorio de usuario en el que está posicionado.
	\item Contexto \textbf{organización}: abre la página de {\it issues} del repositorio de la organización sobre la que está posicionado.
\end{itemize}
\begin{figure}[H]
	\begin{center}
	\includegraphics[width=0.7\textwidth]{images/new-issue}
	\caption{Nuevo issue en repositorio de organización.}
	\label{fig:new-issue}
	\end{center}
\end{figure}

\begin{figure}[H]
	\begin{center}
	\includegraphics[width=0.7\textwidth]{images/issue-form}
	\caption{Formulario de creación de incidencias (issues).}
	\label{fig:new-issue}
	\end{center}
\end{figure}

\subsection{issues}
\label{3.3.8}
Abre el navegador por defecto, situando al usuario en la lista de incidencias ({\it issues}). Debe estar posicionado dentro de un repositorio de \textbf{usuario} o repositorio de \textbf{organización}.

\textbf{Sintaxis}: \verb issues  .

\begin{figure}[H]
	\begin{center}
	\includegraphics[width=0.65\textwidth]{images/list-issues}
	\caption{Ver incidencias de un repositorio.}
	\label{fig:list-issues}
	\end{center}
\end{figure}

\begin{figure}[H]
	\begin{center}
	\includegraphics[width=0.65\textwidth]{images/issues-list}
	\caption{Listado de incidencias del repositorio en GitHub.}
	\label{fig:issues-list}
	\end{center}
\end{figure}

\subsection{files}
\label{3.3.9}

Muestra el contenido y el tipo de contenido  (fichero o directorio) de un repositorio. Debe estar posicionado dentro de un repositorio de \textbf{usuario} o repositorio de \textbf{organización}.
Si no se le proporciona ningún parámetro, muestra el contenido de la raíz del repositorio. Si se le especifica un subdirectorio, se mostrará su contenido.

\textbf{Sintaxis}: \verb files  \verb [subdirectorio]  . 

\begin{figure}[H]
	\begin{center}
	\includegraphics[width=0.5\textwidth]{images/dir-content}
	\caption{Listar contenido del repositorio.}
	\label{fig:dir-content}
	\end{center}
\end{figure}

\begin{figure}[H]
	\begin{center}
	\includegraphics[width=0.5\textwidth]{images/subdir-content}
	\caption{Listar contenido de un subdirectorio.}
	\label{fig:subdir-content}
	\end{center}
\end{figure}

\subsection{commits}
\label{3.3.10}

Muestra los {\it commits} (SHA, fecha, autor y mensaje del {\it commit}) de la rama \verb master , en caso de que no se le pase como parámetro una rama en concreto.
Es necesario estar posicionado sobre un repositorio de cualquiera de los siguientes contextos: \textbf{organización} o \textbf{usuario}.

\textbf{Sintaxis}: \verb commits  \verb [rama_repositorio]  .

\begin{figure}[H]
	\begin{center}
	\includegraphics[width=0.59\textwidth]{images/user-commits}
	\caption{Mostrar commits en un repositorio de usuario.}
	\label{fig:user-commits}
	\end{center}
\end{figure}

\begin{figure}[H]
	\begin{center}
	\includegraphics[width=0.59\textwidth]{images/orgs-commits}
	\caption{Mostrar commits de la rama de un repositorio de organización.}
	\label{fig:user-commits}
	\end{center}
\end{figure}

\subsection{open}
\label{3.3.11}

Abre el navegador por defecto y muestra información de GitHub según el contexto:
\begin{itemize}
	\item Contexto de \textbf{usuario}: si se ejecuta \verb open  a nivel de usuario, se abre en el navegador el perfil GitHub del usuario autenticado en {\it ghedsh}. En caso de estar en un repositorio de usuario, se abre la URL de este repositorio.
	\item Contexto de \textbf{organización}: a nivel de organización abre el perfil de la organización en GitHub. Si el usuario se encuentra posicionado en un repositorio de la organización, abre la URL del repositorio.
	Para este contexto en concreto, es posible pasarle un parámetro que consista en una expresión regular o el nombre de algún miembro y abrir su perfil.
	En este caso, la \textbf{sintaxis} sería: \verb open  \verb "nombre"  (para el nombre del miembro) o bien \verb open  \verb /Regexp/  .
	\item Contexto de \textbf{equipo}: abre la URL del equipo en GitHub.
\end{itemize}

\textbf{Sintaxis} general: \verb open  .

\subsection{orgs}
\label{3.3.12}

Muestra las organizaciones a las que pertenece el usuario autenticado en {\it ghedsh}. Si el comando no recibe ningún parámetro, se mostrarán todas las organizaciones. En caso de proporcionarle un parámetro, debe ser una expresión regular y mostrará los resultados que casen.

Sólo está disponible en contexto de \textbf{usuario}.

\textbf{Sintaxis}: \verb orgs  \verb [/Regexp/]  .

\begin{figure}[H]
	\begin{center}
	\includegraphics[width=0.59\textwidth]{images/show-all-orgs}
	\caption{Mostrar todas las organizaciones del usuario autenticado en ghedsh.}
	\label{fig:show-orgs-regexp}
	\end{center}
\end{figure}

\begin{figure}[H]
	\begin{center}
	\includegraphics[width=0.59\textwidth]{images/show-orgs-regexp}
	\caption{Filtrar organizaciones mediante expresión regular.}
	\label{fig:show-orgs-regexp}
	\end{center}
\end{figure}

\subsection{people}
\label{3.3.13}

Muestra los nombres de los usuarios que componen una organización. Esto incluye miembros y colaboradores externos (si el usuario autenticado en {\it ghedsh} tiene permisos de administrador en la organización).
Si no se le proporciona ningún parámetro, lista todos. En caso de utilizar una expresión regular, se mostrarán los resultados que hayan casado con ella.

Este comando se usa, exclusivamente, cuando el usuario está posicionado dentro del contexto de una \textbf{organización}.

\textbf{Sintaxis}: \verb people  \verb [/Regexp/]  .

\begin{figure}[H]
	\begin{center}
	\includegraphics[width=0.59\textwidth]{images/org-people}
	\caption{Mostrar los miembros de una organización.}
	\label{fig:org-people}
	\end{center}
\end{figure}

\begin{figure}[H]
	\begin{center}
	\includegraphics[width=0.59\textwidth]{images/org-people-regexp}
	\caption{Mostrar mediante expresión regular los miembros de una organización.}
	\label{fig:org-people-regexp}
	\end{center}
\end{figure}

\subsection{teams}
\label{3.3.14}

Muestra los equipos existentes en una organización. Si no se le especifica ningún parámetro, listará todos los equipos. Si se le proporciona una expresión regular, mostrará los resultados que casen con ella.
Se ejecuta, exclusivamente, en el contexto de una \textbf{organización}.

\textbf{Sintaxis}: \verb teams  \verb [/Regexp/]  .

\begin{figure}[H]
	\begin{center}
	\includegraphics[width=0.59\textwidth]{images/org-teams}
	\caption{Listar los equipos de una organización.}
	\label{fig:org-teams}
	\end{center}
\end{figure}

\begin{figure}[H]
	\begin{center}
	\includegraphics[width=0.59\textwidth]{images/org-regexp-teams}
	\caption{Listar mediante expresión regular los equipos de una organización.}
	\label{fig:org-regexp-teams}
	\end{center}
\end{figure}

\subsection{new\_team}
\label{3.3.15}

Crea un nuevo equipo dentro de la organización en la que esté posicionado el usuario. Si el comando no recibe ningún parámetro, se abrirá la URL del formulario para crear el equipo en la web de GitHub. En caso de especificarle un parámetro,
éste tiene que ser un fichero (ver plantilla \cite{B25}) que se encuentre en algún lugar de \verb $HOME  en la máquina local.

Una vez más, este comando sólo se puede ejecutar cuando el usuario está posicionado dentro de una organización en {\it ghedsh}.

\textbf{Sintaxis}: \verb new_team  \verb [ruta_home_fichero]  .

\begin{figure}[H]
	\begin{center}
	\includegraphics[width=0.59\textwidth]{images/create-team-form}
	\caption{Formulario de creación de un equipo en GitHub.}
	\label{fig:create-team-form}
	\end{center}
\end{figure}

\begin{figure}[H]
	\begin{center}
	\includegraphics[width=0.9\textwidth]{images/create-team-file}
	\caption{Creación de un equipo mediante fichero.}
	\label{fig:create-team-file}
	\end{center}
\end{figure}

\subsection{rm\_team}
\label{3.3.16}

Elimina un equipo. No recibe ningún parámetro. Se abre el navegador por defecto y el usuario borrará el o los equipos que desee de la lista proporcionada. Se ejecuta sólo en contexto de \textbf{organización}.

\textbf{Sintaxis}: \verb rm_team  .

\subsection{invite\_outside\_collaborators}
\label{3.3.17}

Invita a ser miembros de la organización a los colaboradores externos de la misma. Si no se especifica ningún parámetro, invita a todos los colaboradores externos. En caso de que reciba un parámetro, tendrá que ser un fichero (ver plantilla \cite{B26}) situado en \verb $HOME  de la máquina local.

Se ejecuta, exclusivamente, cuando el usuario de {\it ghedsh} está situado en el contexto de una \textbf{organización}.

\textbf{Sintaxis}: \verb invite_outside_collaborators  \verb [ruta_home_fichero]  .

\begin{figure}[H]
	\begin{center}
	\includegraphics[width=1\textwidth]{images/invite-collabs}
	\caption{Invitación a ser miembros desde fichero.}
	\label{fig:invite-collabs}
	\end{center}
\end{figure}

\subsection{invite\_member}
\label{3.3.18}

Añade nuevos miembros a una organización. Recibe como parámetros el/los nombres de usuario de GitHub, separados por comas o por espacios.

Sólo puede ejecutarse en el contexto de una \textbf{organización}.

\textbf{Sintaxis}: \verb invite_member  \verb <user1,  \verb user2,  \verb user3,  \verb ...,  \verb n>  .
 
\begin{figure}[H]
	\begin{center}
	\includegraphics[width=1\textwidth]{images/invite-member}
	\caption{Añadir miembros específicos.}
	\label{fig:invite-member}
	\end{center}
\end{figure}

\subsection{invite\_member\_from\_file}
\label{3.3.19}

Añade  nuevos miembros a una organización, especificando su nombre de usuario en GitHub. Recibe como único parámetro un fichero (ver plantilla \cite{B27}) existente en \verb $HOME  de la máquina local. 

El comando se ejecuta, exclusivamente, dentro del contexto de una \textbf{organización}.

\textbf{Sintaxis}: \verb invite_member_from_file  \verb <ruta_home_fichero>  .

\begin{figure}[H]
	\begin{center}
	\includegraphics[width=1\textwidth]{images/add-members-file}
	\caption{Añadir miembros mediante fichero.}
	\label{fig:add-members-fil}
	\end{center}
\end{figure}

%------------------------------------------------------------------------------------------------------------
\section{Comandos que dan soporte al proceso de evaluación}
\label{3:sec:4}

Los comandos que se explicarán en esta sección reflejan uno de los principales objetivos de {\it ghedsh}: aportar funcionalidades específicas
que usen las metodologías de {\it GitHub Education}, facilitando al profesorado la gestión de repositorios del alumnado así como la ejecución de {\it scripts} sobre los mismos.
\bigskip

En este conjunto de comandos tenemos: \verb new_eval , \verb foreach  y \verb foreach_try .

\subsection{new\_eval}
\label{3.4.1}

Permite crear un repositorio de evaluación. Recibe como parámetros el nombre del repositorio de evaluación y una expresión regular que añade como subdirectorios los repositorios que lo conforman. En {\it ghedsh}, un repositorio de evaluación consiste en hacer uso de los submódulos de {\it git}, de manera que se crea un repositorio raíz que contiene como submódulos
todos los proyectos que se van a evaluar. Los pasos que realiza son:
\begin{itemize}
	\item En el directorio actual de la máquina local del usuario, crea un directorio con el mismo nombre del repositorio de evaluación.
	\item Añade como submódulos los repositorios de la organización que casen con la expresión regular.
	\item Ejecuta \verb git  \verb push   y sube el contenido a la plataforma GitHub.
\end{itemize}
\bigskip

Dado que, para comprender todas las ventajas que ofrece este comando, es necesario conocer los submódulos de git ({\it gitsubmodules\cite{B28}}), se explicará en qué consisten a continuación.
\bigskip

Esencialmente, un submódulo es un repositorio que se encuentra contenido dentro de otro repositorio. El submódulo tiene su propio histórico de {\it commits} y, el repositorio raíz que lo contiene, se denomina súper-proyecto o súper-repositorio.
\bigskip

Es probable que, mientras trabajamos en un proyecto, necesitemos usar otro proyecto dentro de él. Quizás se trate de una librería de terceros o una que desarrollamos nosotros mismos de forma separada dentro de un proyecto principal. En estos casos, surge un problema común:
precisamos de ser capaces de tratar los dos proyectos por separado y, aún así, usar uno dentro del otro.
\bigskip

El control de versiones {\it Git} aborda ese problema usando submódulos. Los submódulos permiten mantener un repositorio como un subdirectorio de otro repositorio {\it Git}. Por lo tanto, permite clonar otro repositorio en nuestro proyecto, separando los {\it commits} de cada uno.
\bigskip

En {\it ghedsh}, el comando \verb new_eval  se encuentra disponible únicamente en el contexto de \textbf{organización}.

\textbf{Sintaxis}: \verb new_eval  \verb <nombre_repo_evaluacion>  \verb </Regexp/>  .

\begin{figure}[H]
	\begin{center}
	\includegraphics[width=0.9\textwidth]{images/eval-example}
	\caption{Ejemplo de creación de un repositorio de evaluación.}
	\label{fig:eval-example}
	\end{center}
\end{figure}

\begin{figure}[H]
	\begin{center}
	\includegraphics[width=0.4\textwidth]{images/eval-preview}
	\caption{Estructura de un repositorio de evaluación en GitHub.}
	\label{fig:eval-example}
	\end{center}
\end{figure}

\subsection{foreach}
\label{3.4.2}

Ejecuta sobre cada submódulo el comando especificado como parámetro. Para que el comando lleve a cabo su cometido, se necesita lo siguiente:
\begin{itemize}
	\item Estar en contexto de \textbf{organización} dentro de {\it ghedsh} (único contexto en el que está disponible \verb foreach  ).
	\item Estar posicionado dentro de un repositorio que contenga submódulos, dentro de {\it ghedsh}.
	\item En la máquina local, el directorio actual (donde hemos ejecutado {\it ghedsh}) debe ser el repositorio de evaluación en cuestión.
\end{itemize}

\subsection{foreach\_try}
\label{3.4.3}

\section{Caso de uso}
\label{3:sec:5}



%%%%%%%%%%%%%%%%%%%%%%%%%%%%%%%%%%%%%%%%%%%%%%%%%%%%%%%%%%%%%%%%%%%%%%%%%%%%%%%

\newpage{\pagestyle{empty}}
\thispagestyle{empty}

\chapter{Caso de uso}
\label{chapter:cuatro}

%%%%%%%%%%%%%%%%%%%%%%%%%%%%%%%%%%%%%%%%%%%%%%%%%%%%%%%%%%%%%%%%%%%%%%%%%%%%%
% Chapter 4: Conclusiones y Trabajos Futuros 
%%%%%%%%%%%%%%%%%%%%%%%%%%%%%%%%%%%%%%%%%%%%%%%%%%%%%%%%%%%%%%%%%%%%%%%%%%%%%%%

%++++++++++++++++++++++++++++++++++++++++++++++++++++++++++++++++++++++++++++++

La segunda versión de la gema {\it ghedsh} ha alcanzado satisfactoriamente los objetivos planteados: mejorar su arquitectura inicial, con el fin de facilitar a otros desarrolladores
la incorporación de funcionalidades y, aparte, ser capaz de dar soporte al proceso de evaluación mediante las metodologías de {\it GitHub Education}, ofreciendo funcionalidades que difícilmente
podrán encontrarse actualmente en otros intérpretes de comandos similares. En este sentido, se ha realizado un aporte a la sociedad y, en concreto, a la comunidad educativa.
\bigskip

La retroalimentación es fundamental en el proceso de aprendizaje. Una comunicación eficaz con el alumnado puede marcar la diferencia. Mediante {\it ghedsh}, se puede acceder fácilmente a la interfaz que proporciona {\it GitHub} para comunicarse con los estudiantes, la cual permite comentar líneas y porciones de código en las asignaciones, logrando
mejorar cualitativamente las indicaciones del profesor. Ésto favorece que el estudiantado se realice preguntas acerca del código y comprender cómo es su funcionamiento.
\bigskip

Por otro lado, está previsto que el desarrollo de {\it ghedsh} tenga continuidad en el tiempo, puesto que, a pesar de que su diseño se ha optimizado notablemente, aún existe margen de mejora, lo que aumenta incluso más el potencial de esta herramienta.
\bigskip

Como desarrollo de líneas futuras, se pueden tener en cuenta diversos aspectos:
\begin{itemize}
	\item Orientar el diseño para cumplir totalmente el principio {\it Open/Closed} (OCP). Ésto lograría que {\it ghedsh} fuera extensible en funcionalidad mediante {\it plugins} desarrollados por otros programadores.
	\item Ofrecer una ayuda completa y estructurada dentro de la herramienta.
	\item Añadir información extendida de los alumnos, de manera que se pueda ver, por ejemplo, su perfil en el Campus Virtual.
	\item Mejorar la estructura de pruebas de la herramienta.
	\item Enriquecer los comandos que dan soporte al proceso de evaluación, mediante un conjunto de librerías que realicen tareas especializadas para este proceso.
\end{itemize}





%%%%%%%%%%%%%%%%%%%%%%%%%%%%%%%%%%%%%%%%%%%%%%%%%%%%%%%%%%%%%%%%%%%%%%%%%%%%%%%

\newpage{\pagestyle{empty}}
\thispagestyle{empty}

\chapter{Conclusiones y líneas futuras}
\label{chapter:Conclusiones}

%%%%%%%%%%%%%%%%%%%%%%%%%%%%%%%%%%%%%%%%%%%%%%%%%%%%%%%%%%%%%%%%%%%%%%%%%%%%%
% Chapter 5: Summary and Conlusions
%%%%%%%%%%%%%%%%%%%%%%%%%%%%%%%%%%%%%%%%%%%%%%%%%%%%%%%%%%%%%%%%%%%%%%%%%%%%%%%

%++++++++++++++++++++++++++++++++++++++++++++++++++++++++++++++++++++++++++++++
The second version of the ghedsh gem has succesfully achieved the objectives outlined
before: improve its initial architecture to prepare it for changes and enable
incorporating future functionalities. Moreover, this will give support to the evaluation
process through GitHub Education’s methodologies. The improvement will offer
funcionalities that will most certainly be difficult to find in others shell prompts.
This has significant implications and contributes to the educational community.
\bigskip

Feedback is key in the learning process. Thus, an effective and efficient communication
with students can make a real difference. With ghedsh it is easier to access the interface
provided by GitHub, which allows to comment lines and portions of the code in the
respective allocations. Hence, the teacher corrections are qualitatively enhanced. In
this way, students can easliy ask questions about the code, but what is more, to
understand how it works.
\bigskip

Additionally, it is planned that ghedsh will be subject to continued further
development as it has been optimised. However, there is still room for improvement
which highly increases the potential of such tool.
\bigskip

For future work lines, some aspects may be considered:
\begin{itemize}
  \item Guide the design towards Open/Close (OCP) principle. Therefore ghedsh will
  be more extensible (in terms of functionality) including plug-ins developed by
  others.
  \item Provide fully structured usage help within the tool.
  \item Include extra information about students, so their profile in “Campus virtual”
  can be visible.
  \item Improve the tool current test structure.
  \item Add extra functionality to the commands that support the evaluation process, by including a set of libraries
  that perform specific tasks for this process.
\end{itemize}




%%%%%%%%%%%%%%%%%%%%%%%%%%%%%%%%%%%%%%%%%%%%%%%%%%%%%%%%%%%%%%%%%%%%%%%%%%%%%%%
\newpage{\pagestyle{empty}}
\thispagestyle{empty}

\chapter{Summary and Conclusions}
\label{chapter:ingles}

%%%%%%%%%%%%%%%%%%%%%%%%%%%%%%%%%%%%%%%%%%%%%%%%%%%%%%%%%%%%%%%%%%%%%%%%%%%%%
% Chapter 6: Presupuesto
%%%%%%%%%%%%%%%%%%%%%%%%%%%%%%%%%%%%%%%%%%%%%%%%%%%%%%%%%%%%%%%%%%%%%%%%%%%%%%%

%++++++++++++++++++++++++++++++++++++++++++++++++++++++++++++++++++++++++++++++


En este capítulo se desglosará el coste que supondría realizar este Trabajo de Fin de Grado para un cliente real.
\bigskip

En este caso, se tomará como consideración que el precio por hora es de 30\euro{}/hora. Cada una de las siguientes tablas especifica la duración de la actividad y el subtotal.

\section{Comandos del núcleo en ghedsh}
\label{6:sec:1}
%--------------------------------------------------------------------------
\begin{table}[!ht]
\begin{center}
\begin{tabular}{|p{80mm}|p{25mm}|p{20mm}|} \hline 
\textbf{Actividad} & \textbf{Duración} & \textbf{Precio} \\ \hline

Desarrollo de los comandos del núcleo de ghedsh &
10 horas &
300 \euro{}
\\
\hline
\hline \hline
{\bfseries Subtotal} &
{\bfseries 10 horas} &
{\bfseries 300 \euro{}}
\\
\hline

\end{tabular}
\end{center}
\caption{Tabla de actividades, duración y precios de los comandos del núcleo de ghedsh.}
\label{table:resOthers1}
\end{table}


%---------------------------------------------------------------------------------
\section{Comandos incorporados de ghedsh}
\label{6:sec:2}
%--------------------------------------------------------------------------
\begin{table}[!ht]
\begin{center}
\begin{tabular}{|p{80mm}|p{25mm}|p{20mm}|} \hline 
\textbf{Actividad} & \textbf{Duración} & \textbf{Precio} \\ \hline

Desarrollo de los comandos incorporados &
200 horas &
6000 \euro{}
\\
\hline
\hline \hline
{\bfseries Subtotal} &
{\bfseries 200 horas} &
{\bfseries 6000 \euro{}}
\\
\hline

\end{tabular}
\end{center}
\caption{Tabla de actividades, duración y precios de los comandos incorporados por ghedsh.}
\label{table:resOthers2}
\end{table}


%---------------------------------------------------------------------------------
\section{Comandos que dan soporte al proceso de evaluación}
\label{6:sec:3}
%--------------------------------------------------------------------------
\begin{table}[!ht]
  \begin{center}
  \begin{tabular}{|p{80mm}|p{25mm}|p{20mm}|} \hline 
  \textbf{Actividad} & \textbf{Duración} & \textbf{Precio} \\ \hline
  
  Desarrollo de los comandos que dan soporte al proceso de evaluación &
  25 horas &
  750 \euro{}
  \\
  \hline
  \hline \hline
  {\bfseries Subtotal} &
  {\bfseries 25 horas} &
  {\bfseries 750 \euro{}}
  \\
  \hline
  
  \end{tabular}
  \end{center}
  \caption{Tabla de actividades, duración y precios de los comandos que dan soporte al proceso de evaluación.}
  \label{table:resOthers3}
  \end{table}
  
  

%---------------------------------------------------------------------------------

\section{Coste y duración total}
\label{6:sec:4}
%--------------------------------------------------------------------------
\begin{table}[!ht]
\begin{center}
\begin{tabular}{|p{80mm}|p{25mm}|p{20mm}|} \hline 
\textbf{Actividad} & \textbf{Duración} & \textbf{Precio} \\ \hline

Desarrollo de los comandos del núcleo de ghedsh &
10 horas &
300 \euro{}
\\
\hline

Desarrollo de los comandos incorporados en ghedsh &
200 horas &
6000 \euro{}
\\
\hline
Desarrollo de los comandos que dan soporte al proceso de evaluación &
25 horas &
750 \euro{}
\\
\hline
\hline \hline
{\bfseries Total} &
{\bfseries 235 horas} &
{\bfseries 7050 \euro{}}
\\
\hline

\end{tabular}
\end{center}
\caption{Precio y duración total}
\label{table:resOthers3}
\end{table}



%---------------------------------------------------------------------------------



%%%%%%%%%%%%%%%%%%%%%%%%%%%%%%%%%%%%%%%%%%%%%%%%%%%%%%%%%%%%%%%%%%%%%%%%%%%%%%%
\newpage{\pagestyle{empty}}
\thispagestyle{empty}

\chapter{Presupuesto}
\label{chapter:Presupuesto}

\input{cap7.tex}

%%%%%%%%%%%%%%%%%%%%%%%%%%%%%%%%%%%%%%%%%%%%%%%%%%%%%%%%%%%%%%%%%%%%%%%%%%%%%%%

%%%%%%%%%%%%%%%%%%%%%%%%%%%%%%%%%%%%%%%%%%%%%%%%%%%%%%%%%%%%%%%%%%%%%%%%%%%%%%%
\newpage{\pagestyle{empty}}
\thispagestyle{empty}
\begin{appendix}

\chapter{Repositorios}
\label{appendix:1}
\input{apendice1.tex}

\end{appendix}

%%%%%%%%%%%%%%%%%%%%%%%%%%%%%%%%%%%%%%%%%%%%%%%%%%%%%%%%%%%%%%%%%%%%%%%%%%%%%%%
\addcontentsline{toc}{chapter}{Bibliografía}
\bibliographystyle{plain}

\bibliography{memtfg}
\nocite{*}

%%%%%%%%%%%%%%%%%%%%%%%%%%%%%%%%%%%%%%%%%%%%%%%%%%%%%%%%%%%%%%%%%%%%%%%%%%%%%%%

\end{document}
