%%%%%%%%%%%%%%%%%%%%%%%%%%%%%%%%%%%%%%%%%%%%%%%%%%%%%%%%%%%%%%%%%%%%%%%%%%%%%%%
% Chapter 2: Desarrollo
%%%%%%%%%%%%%%%%%%%%%%%%%%%%%%%%%%%%%%%%%%%%%%%%%%%%%%%%%%%%%%%%%%%%%%%%%%%%%%%

%++++++++++++++++++++++++++++++++++++++++++++++++++++++++++++++++++++++++++++++
En este capítulo dos se va a describir el desarrollo del proyecto. Se dividirá en dos fases bien diferenciadas, una primera fase que consiste en el análisis y refactorización del código fuente de la versión anterior de
{\it GitHub Education Shell}, y una segunda fase que trata de la incorporación de nuevas funcionalidades a la herramienta.
\bigskip

Por otro lado, también cabe nombrar la metodología de trabajo empleada. Situándonos en el marco de las metodologías ágiles, {\it Scrum} fue la metodología que mejor encajaba, teniendo en cuenta las características del proyecto.
Se ha optado por un desarrollo incremental, en lugar de una planificación y ejecución estricta de las tareas. Además, en numerosas ocasiones,
se produjeron solapamientos de las diferentes partes del desarrollo, en vez de un ciclo secuencial. También fueron frecuentes las reuniones con el tutor, en las que se comentaban tanto avances como dificultades.

%++++++++++++++++++++++++++++++++++++++++++++++++++++++++++++++++++++++++++++++

\section{Primera fase: análisis y refactorización.}
\label{2:sec:1}

En esta sección, se explicará detalladamente el proceso fundamental de la primera fase de este Trabajo de Fin de Grado.


%---------------------------------------------------------------------------------
\subsection{Análisis}
\label{subsec:2.1.1}

El análisis del código fuente correspondiente a la primera versión de {\it ghedsh}, se ha llevado a cabo con la finalidad de identificar aquellas partes mejorables del diseño e implementación,
puesto que una de las prioridades es facilitar el desarrollo colaborativo y, para ello, se requiere que el código sea limpio y fácil de entender, lo más auto-explicativo posible.

\begin{figure}[H]
\begin{center}
%\includegraphics[width=0.47\textwidth]{images/}
\caption{Ramas del repositorio}
%\label{fig:github2}
\end{center}
\end{figure}


La documentación adicional para llevar a cabo los desarrollos de cada iteración, así como los problemas detectados, se anotaban en el apartado de \verb1issues1 con el fin de que quedara constancia de ello y se reflejara el estado en el que se encontraba cada uno.

\begin{figure}[H]
\begin{center}
%\includegraphics[width=1\textwidth]{images/}
\caption{Apartado de issues}
%\label{fig:github3}
\end{center}
\end{figure}

%---------------------------------------------------------------------------------
\subsection{Travis-CI}
\label{subsec:2.1.2}

Como herramienta de integración continua, se ha utilizado Travis-CI, con el fin de asegurarnos el despliegue de la aplicación era satisfactorio tras cada cambio subido a la herramienta de control de versiones (GitHub).

%---------------------------------------------------------------------------------
\subsection{Experiencia de usuario}
\label{subsec:2.1.3}

Por otra parte, el tutor del Trabajo de Fin de Máster ha hecho pruebas reales con el resultado de cada iteración, actuando como {\it Product Owner}. De este modo, se comprobaba el funcionamiento de la aplicación en un entorno real y se recibía un valioso feedback para corregir problemas o hacer mejoras en las siguientes iteraciones.
