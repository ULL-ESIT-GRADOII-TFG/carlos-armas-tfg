%%%%%%%%%%%%%%%%%%%%%%%%%%%%%%%%%%%%%%%%%%%%%%%%%%%%%%%%%%%%%%%%%%%%%%%%%%%%%
% Chapter 1: Introducción 
%%%%%%%%%%%%%%%%%%%%%%%%%%%%%%%%%%%%%%%%%%%%%%%%%%%%%%%%%%%%%%%%%%%%%%%%%%%%%%%

%---------------------------------------------------------------------------------
\section{Antecedentes}
\label{1:sec:1}

Git es un software de control de versiones de código abierto ampliamente utilizado, diseñado por Linus Torvalds.
El propósito del control de versiones es llevar un registro de los diversos cambios que se realizan sobre
los elementos de un proyecto y, de esta manera, facilitar la coordinación del trabajo que varias personas
realizan sobre el mismo.
\bigskip

Por otro lado, GitHub es una plataforma de desarrollo colaborativo empleada para alojar proyectos que utilizan el control de verisones Git.
Esta plataforma ofrece numerosos servicios que dan soporte al trabajo colaborativo. Entre los más destacados tenemos: repositorios, documentación,
gestión de incidencias, gestión de equipos, chats, notificaciones, alojamiento de páginas web y tableros para organizar de manera flexible y visual las tareas que componen un proyecto.
\bigskip

Asimismo, GitHub dispone de un programa denominado {\it GitHub Education} que ofrece una variedad de herramientas tanto para el profesorado como para estudiantes.
\bigskip

A los docentes se les ofrece la posibilidad de crear organizaciones para cada clase, mediante repositorios privados que se crean automáticamente para el conjunto de estudiantes y equipos,
a partir de las asignaciones realizadas por los profesores de las asignaturas a través la herramienta {\it GitHub Classroom }.
\bigskip

En cuanto a los estudiantes, se les da acceso gratuito a herramientas de desarrollo, plataformas de alojamiento y nuevas características en GitHub gracias a {\it Student Developer Pack }.
\bigskip

En este contexto, multitud de docentes se plantean la necesidad del desarrollo de metodologías e implantación de herramientas
que asistan al profesorado en la evaluación de las tareas y actividades de programación, principalmente.
%---------------------------------------------------------------------------------
\section{Estado actual del arte}
\label{1:sec:2}
Con el fin de integrar el uso del control de versiones en las aulas, GitHub desarrolló una herramienta de línea de comandos bajo el nombre de {\it Teachers Pet}. Ésta herramienta proponía que cada aula fuese una
organización, lo que permitía a los profesores (que serían los propietarios de la organización) administrar todos los repositorios dentro de la misma. Con esta forma de trabajar se lograban dos objetivos:
\begin{itemize}
  \item Facilitar a los profesores la distribución de código de inicio a los alumnos.
  \item Los profesores eran capaces de acceder a las tareas de los alumnos para resolver preguntas y comprobar el progreso.
\end{itemize}
Sin embargo, cayó en desuso y se dejó de desarrollar. La principal razón fue que los comandos se hacían excesivamente largos en determinadas tareas, puesto que era necesario especificar múltiples opciones.
\bigskip

Por otro lado, también existen diversas alternativas desarrolladas por la comunidad. Éstas son:
\begin{itemize}
  \item {\it ghi} \cite{B1} (GitHub Issues): proporciona una manera fácil de gestionar las incidencias ({\it issues}) de los repositorios desde la terminal del usuario, utilizando su editor de preferencia. Es posible llevar a cabo diversas acciones como, por ejemplo,
  listar incidencias, abrirlas y cerrarlas y comentar en ellas, entre otras.
  \item {\it ghs} \cite{B2} (GitHub Search): es una utilidad que permite realizar búsquedas de repositorios alojados en GitHub. Admite diversas opciones para limitar las búsquedas en el ámbito deseado, como organizaciones y usuarios, por ejemplo.
\end{itemize}

Actualmente, GitHub dispone de una plataforma, {\it GitHub Classroom} \cite{B3}, que mejora la experiencia de usuario respecto a las herramientas anteriormente nombradas, dado que dispone de una interfaz amigable en el navegador que simplifica
la configuración de las aulas y asignaciones. No obstante, en esta plataforma existen algunas limitaciones:
\begin{itemize}
  \item No dispone de alguna funcionalidad dentro de ella que permita al profesor crear un repositorio de evaluación para calificar las tareas.
  \item No da soporte a herramientas de integración continua, como {\it Travis CI}, {\it CircleCI}, {\it Jenkins}, etc.
  \item En muchos casos, el nombre de usuario que ha escogido el alumno al registrarse en GitHub no permite identificarlo. Además, el sistema para añadir información adicional del alumno es incómodo de usar,
  puesto que la información se inserta individualmente.
\end{itemize}
%---------------------------------------------------------------------------------
\section{Objetivos y actividades a realizar}
\label{1:sec:3}

El principal objetivo de este Trabajo de Fin de Grado es desarrollar una interfaz de línea de comandos, bajo el nombre de {\it GitHub Education Shell},
que utilice la misma metodología que {\it Teachers Pet} y {\it GitHub Classroom} y que aporte una solución a las limitaciones que poseen dichas herramientas, así como priorizar
la realización de tareas a gran escala más comunes en entornos educativos.
\bigskip

Para lograr dichos objetivos, se han definido diferentes actividades:
\begin{itemize}
  \item {\bf A1.} Estudio del funcionamiento de la API ({\it Application Programming Interface}) de GitHub y su implementación oficial en Ruby, {\it Octokit}.
  \item {\bf A2.} Analizar aplicaciones similares e identificar aspectos a mejorar.
  \item {\bf A3.} Comprensión del código de la última versión de la gema {\it ghedsh}.
  \item {\bf A4.} Estudio de aspectos positivos y negativos del diseño de dicha gema.
  \item {\bf A5.} Refactorización del código fuente, aplicando patrones de diseño.
  \item {\bf A6.} Estudio de nuevas funcionalidades para incorporar.
  \item {\bf A7.} Implementación de las funcionalidades escogidas.
  \item {\bf A8.} Definir estructura básica de pruebas.
  \item {\bf A9.} Documentación del código.
\end{itemize}
\bigskip

En la siguiente tabla se muestra el plan de trabajo con la duración de las actividades:
%--------------------------------------------------------------------------
\begin{table}[!ht]
\begin{center}
\begin{tabular}{|p{25mm}|p{50mm}|} \hline 
\textbf{Objetivo} & \textbf{Fecha} \\ \hline

A1, A2, A3 & Febrero \\
\hline

A4, A5 & Marzo, Abril 
\\
\hline

A6, A7 & Mayo
\\
\hline

A8, A9 & Junio
\\
\hline

\end{tabular}
\end{center}
\caption{Tabla resumen del plan de trabajo}
\label{table:resOthers}
\end{table}

%---------------------------------------------------------------------------------
\section{Tecnología empleada}
\label{1:sec:4}
 
En cuanto a la tecnología empleada, el lenguaje de programación escogido para el desarrollo ha sido {\it Ruby},
puesto que la versión anterior también está escrita en este lenguaje.
\bigskip

Para que el programa utilice los datos de GitHub del usuario, se ha utilizado la API REST v3. En concreto,
la librería oficial escrita en Ruby, \cite{B4}, la cual proporciona una gran cantidad de métodos para realizar diferentes acciones, como, por ejemplo,
crear repositorios y administrar su configuración, crear equipos y acceder a la información del usuario, entre otros.
Para la autenticación del usuario, se generará un {\it token} automáticamente con los permisos requeridos para el uso del programa.

Por otro lado, en cuanto al ecosistema de {\it Ruby}, tenemos:
\begin{itemize}
  \item {\it Rubygems} \cite{B5}: servicio de alojamiento de gemas para la comunidad de Ruby.
  \item {\it Bundler} \cite{B6}: proporciona un entorno para manejar las dependencias de un proyecto Ruby, añadiendo las versiones que son necesarias.
  \item {\it YARD} \cite{B7}: herramienta para la documentción del código fuente.
  \item {\it RSpec} \cite{B8}: librería para la definición de pruebas.
\end{itemize}
