\documentclass{beamer}
%\documentclass[xcolor=dvipsnames]{beamer}
\usepackage[spanish]{babel}
\usepackage[utf8]{inputenc}
\usepackage{graphicx}
\usepackage{latexsym}

\newcommand{\beamer}{\textsc{beamer}}
\newtheorem{definicion}{Definición}
\newtheorem{ejemplo}{Ejemplo}

%%%%%%%%%%%%%%%%%%%%%%%%%%%%%%%%%%%%%%%%%%%%%%%%%%%%%%%%%%%%%%%%%%%%%%%%%%%%%%%
\title[Trabajo de Fin de Grado]{ghedsh: Un intérprete de
comandos para GitHub Education\\
GitHub Education Shell: ghedsh.}

\author[Carlos de Armas Hernández] {
Autor: Carlos de Armas Hernández \\
Director: Casiano Rodríguez León
}

\institute[ULL]{Escuela Superior de Ingeniería y Tecnología \\
                Departamento de Ingeniería Informática y de Sistemas \\
                Universidad de La Laguna}
\date[12-07-2018]{12 de julio de 2018}
%%%%%%%%%%%%%%%%%%%%%%%%%%%%%%%%%%%%%%%%%%%%%%%%%%%%%%%%%%%%%%%%%%%%%%%%%%%%%%%

%\usetheme{Berlin}
\usetheme{Madrid}

%%%%%%%%%%%%%%%%%%%%%%%%%%%%%%%%%%%%%%%%%%%%%%%%%%%%%%%%%%%%%%%%%%%%%%%%%%%%%%%
\definecolor{pantone254}{RGB}{92,6,140}
\definecolor{pantone3015}{RGB}{92,6,140}
\definecolor{pantone432}{RGB}{92,6,140}
\setbeamercolor*{palette primary}{use=structure,fg=white,bg=pantone254}
\setbeamercolor*{palette secondary}{use=structure,fg=white,bg=pantone3015}
\setbeamercolor*{palette tertiary}{use=structure,fg=white,bg=pantone432}
\setbeamercolor*{palette sidebar primary}{use=structure,fg=pantone254}
\setbeamercolor*{palette sidebar tertiary}{use=structure,fg=pantone3015}
\setbeamercolor*{block title}{bg=pantone3015,fg=white}
\setbeamercolor*{alerted text}{fg=pantone432}
\setbeamercolor*{item projected}{fg=pantone254}
\setbeamercolor*{section in toc shaded}{use=structure,fg=structure.fg}
\setbeamercolor*{section in toc}{fg=pantone3015}
\setbeamercolor*{subsection in toc shaded}{fg=pantone3015}
\setbeamercolor*{subsection in toc}{fg=pantone432}

%%%%%%%%%%%%%%%%%%%%%%%%%%%%%%%%%%%%%%%%%%%%%%%%%%%%%%%%%%%%%%%%%%%%%%%%%%%%%%%
\begin{document}
  
%++++++++++++++++++++++++++++++++++++++++++++++++++++++++++++++++++++++++++++++  
\begin{frame}

  \includegraphics[width=0.3\textwidth]{img/ull.eps}
  \hspace*{7.5cm}
  \titlepage

\end{frame}
%++++++++++++++++++++++++++++++++++++++++++++++++++++++++++++++++++++++++++++++  

%++++++++++++++++++++++++++++++++++++++++++++++++++++++++++++++++++++++++++++++  
\begin{frame}
  \frametitle{Índice}  
  \tableofcontents
\end{frame}
%++++++++++++++++++++++++++++++++++++++++++++++++++++++++++++++++++++++++++++++  

\section{Introducción}
\begin{frame}[allowframebreaks,fragile]
  \frametitle{Introducción}
  \textbf{¿Qué es ``ghedsh''?}
  \bigskip

  Es una gema Ruby que consiste en un intérprete de comandos desarrollado para integrar las metodologías
  de GitHub Education, viendo las organizaciones como aulas y los repositorios como las asignaciones de los alumnos.
  
  \begin{center}
    \begin{figure}[!htb]  
      \minipage{0.5\textwidth}%
        \includegraphics[width=\linewidth]{img/ghedsh-logo.png}
      \endminipage
    \end{figure}
  \end{center}

  \framebreak
  %+++++++++++++++++++++++++++++++++++++++++++++++++++++++++++++++++++++++++++++++++++++++++++++++++++++++++++++++++++++++++++++++++++++++++++
  
  En cuanto a herramientas similares, existen las siguientes:
  \begin{itemize}
    \item {\it Teachers Pet}.
    \item {\it ghi} (GitHub Issues).
    \item {\it ghs} (GitHub Search).
  \end{itemize}

  \framebreak
  Desarrolladas por {\it GitHub}:
  \begin{itemize}
    \item \textbf{Teachers Pet}. {\it CLI} desarrollado previamente a {\it Classroom}.
    \begin{itemize}
      \item Inconveniente: los comandos se hacían excesivamente largos en determinados casos. 
      Cayó en desuso y se dejó de desarrollar.
    \end{itemize}
  \end{itemize}

  Desarrolladas por la comunidad:
  \begin{itemize}
    \item \textbf{ghi} (GitHub Issues). Permite gestionar SOLO las incidencias (issues) de los repositorios desde la terminal del usuario.
    \item \textbf{ghs} (GitHub Search). Permite realizar búsquedas de repositorios alojados en GitHub. Es poco ágil.
  \end{itemize}
\end{frame}
  %+++++++++++++++++++++++++++++++++++++++++++++++++++++++++++++++++++++++++++++++++++++++++++++++++++++++++++++++++++++++++++++++++++++++++++

\section{Objetivos}
\begin{frame}[fragile]
  \frametitle{Objetivos}
  
  \begin{itemize}
    \item Esta segunda versión de {\it ghedsh} busca mejorar el código fuente de la primera versión,
  teniendo en cuenta aspectos como:
  \begin{itemize}
    \item La mantenibilidad del código 
    \item Facilitar la incorporación de nuevas funcionalidades por parte de terceros.
  \end{itemize}
    \item Por otro lado, se han añadido funcionalidades que dan soporte al proceso de evaluación.
  \end{itemize}
\end{frame}

%++++++++++++++++++++++++++++++++++++++++++++++++++++++++++++++++++++++++++++++  

\section{Tecnologías empleadas}
\begin{frame}[fragile]
  \frametitle{Tecnologías empleadas}
  \begin{columns}
    \column{0.3\linewidth}
       \centering
       \includegraphics[height=1.2cm, width=1.2cm]{img/ruby-logo.png}
    \column{0.6\linewidth}
        Lenguaje de programación: \textbf{Ruby}
  \end{columns}
  \bigskip

  \begin{columns}
    \column{0.3\linewidth}
       \centering
       \includegraphics[height=1cm, width=3.5cm]{img/bundler-logo.png}
    \column{0.6\linewidth}
        Gestión de dependencias: \textbf{Bundler}
  \end{columns}
  \bigskip

  \begin{columns}
    \column{0.3\linewidth}
       \centering
       \includegraphics[height=2cm, width=2cm]{img/octokit-logo.png}
    \column{0.6\linewidth}
        API de GitHub: \textbf{octokit}
  \end{columns}
  \bigskip

  \begin{columns}
    \column{0.3\linewidth}
       \centering
       \includegraphics[height=1.2cm, width=1.2cm]{img/rspec-logo.png}
    \column{0.6\linewidth}
        Testing: \textbf{RSpec}
  \end{columns}
\end{frame}
%++++++++++++++++++++++++++++++++++++++++++++++++++++++++++++++++++++++++++++++

\section{Desarrollo del proyecto}
\begin{frame}
\frametitle{Desarrollo del proyecto}
  Dividimos el desarrollo del proyecto en dos fases bien diferenciadas:
  \begin{itemize}
    \item Análisis. Identificar aquellas partes mejorables del diseño e implementación iniciales.
    \item Refactorización. Proceso llevado a cabo para solucionar las debilidades anteriores.
  \end{itemize}
\end{frame}

\subsection{Primera fase. Análisis}
\begin{frame}[fragile]
\frametitle{Primera fase. Análisis}
  Tras estudiar el código de la primera versión de la gema se han detectado diversos {\it code smell}.
  \bigskip

  \textbf{¿Qué es un code smell?}
  \bigskip

  Se define como cualquier característica del código fuente que, posiblemente, 
  indica un problema más profundo. No son considerados como bugs.
\end{frame}

\begin{frame}
\frametitle{Primera fase. Análisis}
\framesubtitle{Switch Smell}

  \begin{columns}
    \column{0.2\linewidth}
      \includegraphics[height=2.5cm, width=2cm]{img/constantes.png}
    \column{0.6\linewidth}
    \includegraphics[height=7cm, width=7cm]{img/switch-smell.png}
  \end{columns}

\end{frame}

\begin{frame}
  \frametitle{Primera fase. Análisis}
  \framesubtitle{Long Method}
  {\it Long Method} se clasifica a nivel de método. Como su propio nombre indica,
  consiste en un método que ha crecido demasiado y dificulta saber qué es lo que realmente hace.
\end{frame}

\begin{frame}
  \frametitle{Primera fase. Análisis}
  \framesubtitle{Large Class}
    Large Class se clasifica dentro de los smells a nivel de clases. Indica que una clase ha crecido excesivamente en tamaño
    (God Object). Su funcionalidad puede descomponerse en clases más pequeñas.
\end{frame}


  %+++++++++++++++++++++++++++++++++++++++++++++++++++++++++++++++++++++++++++++++++++++++++++++++++++++++++
\subsection{Segunda fase. Refactorización}
\begin{frame}
\frametitle{Segunda fase. Refactorización}
  Esta fase se centra en eliminar las debilidades anteriormente comentadas. Gran parte de este proceso
  consistió en eliminar el {\it Switch Smell}, ya que era el más repetido a lo largo del código fuente.

\end{frame}

%+++++++++++++++++++++++++++++++++++++++++++++++++++++++++++++++++++++++++++++++++++++++++++++++++++++++++++
\begin{frame}
  \frametitle{Segunda fase. Refactorización}
  \framesubtitle{Strategy Pattern}
  Para eliminar el {\it Switch Smell} hemos aplicado el Patrón Estrategia ({\it Strategy Pattern}).
  \bigskip

  El propósito de este patrón es proporcionar una manera clara de definir familias de algoritmos y poder intercambiarlos fácilmente.
  
\end{frame}

\begin{frame}
  \frametitle{Segunda fase. Refactorización}
  \framesubtitle{Strategy Pattern}

  \centering
  \includegraphics[height=7cm, width=7cm]{img/new-loop.png}
\end{frame}

\begin{frame}
  \frametitle{Segunda fase. Refactorización}
  \framesubtitle{Extract Method}

  \centering
  \includegraphics[height=3cm, width=10cm]{img/extract-method-example.png}
\end{frame}

\begin{frame}
  \frametitle{Segunda fase. Refactorización}
  \framesubtitle{Extract Class}

  Una clase debería tener una única responsabilidad.
  \bigskip

  \begin{itemize}
    \item Paso 1. Determinar qué se va a extraer.
    \item Paso 2. Crear una nueva clase.
    \item Paso 3. Renombrar la clase antigua.
  \end{itemize}
\end{frame}

%+++++++++++++++++++++++++++++++++++++++++++++++++++++++++++++++++++++++++++++++++++++++++++++++++++++++++++
\section{Resultados obtenidos}
\begin{frame}
\frametitle{Resultados obtenidos}
  Tras la etapa de desarrollo, {\it GitHub Education Shell} incorpora las siguientes características:
  \begin{itemize}
    \item Autenticación con credenciales de GitHub (OAuth).
    \item Conjunto de comandos
    \begin{itemize}
      \item Comandos del núcleo de {\it ghedsh}.
      \item Comandos incorporados ({\it built-in commands}).
      \item Comandos que dan soporte al proceso de evaluación.
    \end{itemize}
  \end{itemize}
\end{frame}

\begin{frame}
  \frametitle{Resultados obtenidos}
  \framesubtitle{Comandos del núcleo de {\it ghedsh}}

  \begin{itemize}
    \item \textbf{cd}: permite movernos entre repositorios, organizaciones, equipos, etc.
    \item \textbf{! ó bash}: interpreta la entrada del usuario como un comando tipo Unix.
  \end{itemize}

\end{frame}

\begin{frame}
  \frametitle{Resultados obtenidos}
  \framesubtitle{Comandos incorporados}

  \begin{minipage}[t]{0.48\linewidth}
    \centering
    \includegraphics[height=4cm, width=4cm]{img/commands2.png}
  \end{minipage}%
  \begin{minipage}[t]{0.48\linewidth}
    \includegraphics[height=4cm, width=1.5cm]{img/commands1.png}
  \end{minipage}

\end{frame}

\begin{frame}
  \frametitle{Resultados obtenidos}
  \framesubtitle{Comandos que dan soporte al proceso de evaluación}

  Tenemos los siguientes:
    \begin{itemize}
      \item \textbf{new\_eval}
      \item \textbf{foreach}
      \item \textbf{foreach\_try}
    \end{itemize}
\end{frame}

\begin{frame}
  \frametitle{Resultados obtenidos}
  \framesubtitle{new\_eval}
  \textbf{new\_eval}
  \bigskip

  Permite crear un repositorio de evaluación. Consiste en hacer uso de los submódulos de {\it git},
  de manera que se crea un ``súper repositorio'' que contiene como submódulos los proyectos que se van a evaluar.
  \bigskip

  \centering
  \includegraphics[height=3cm, width=4.5cm]{img/eval-preview.png}

\end{frame}

\begin{frame}[fragile]
  \frametitle{Resultados obtenidos}
  \framesubtitle{foreach y foreach\_try}
  \textbf{foreach}
  \bigskip

  Ejecuta para cada submódulo el comando especificado (por ejemplo, \verb git  \verb pull ). Internamente realiza \verb git  \verb submodule  \verb foreach .
  Si ocurre algún error en un submódulo durante la ejecución del comando, pasa al siguiente submódulo.
  \bigskip

  \textbf{foreach\_try}
  \bigskip

  Realiza lo mismo que \verb foreach . Sin embargo, \verb foreach_try  dentendrá su ejecución si ocurre
  algún error en un submódulo.

\end{frame}

\begin{frame}[fragile]
  \frametitle{Resultados obtenidos}
  \framesubtitle{foreach y foreach\_try}
  
  Hagámonos una pregunta:
  \bigskip

  \textbf{¿Y si en lugar de ejecutar un comando simple, hacemos uso de una librería o paquete que realiza tareas más complejas?}

\end{frame}

\begin{frame}[fragile]
  \frametitle{Resultados obtenidos}
  \framesubtitle{ghedsh-grade-node}
  
  Aparte de {\it ghedsh}, hemos desarrollado un paquete {\it npm} llamado {\it ghedsh-grade-node}
  que realiza lo siguiente:
  \begin{itemize}
    \item Recibe como argumento el directorio donde se encuentran las pruebas privadas que ha escrito el profesor.
    \item Copia en el subdirectorio \verb /test  del alumno las pruebas del profesor.
    \item Realiza \verb npm  \verb install .
    \item Realiza \verb npm  \verb test , redirigiendo la salida (tanto \verb stdout  como \verb stderr ) a un fichero.
  \end{itemize}

  \centering
  \includegraphics[height=3cm, width=3cm]{img/ghedsh-grade-node.png}

\end{frame}
%+++++++++++++++++++++++++++++++++++++++++++++++++++++++++++++++++++++++++++++++++++++++++++++++++++++++++++
  
\section{Caso de uso}

\begin{frame}[fragile]
\frametitle{Caso de uso}

\end{frame}

%+++++++++++++++++++++++++++++++++++++++++++++++++++++++++++++++++++++++++++++++++++++++++++++++++++++++++++

\section{Conclusions and future work lines}
\begin{frame}[allowframebreaks]
  \frametitle{Conclusions and future work lines}
  The second version of {\it GitHub Education Shell} has successfully achieved the objectives outlined before:
  \begin{itemize}
    \item  improve its initial architecture and enable incorporating future functionalities.
    \item  also, provide support to the evaluation process through GitHub Education’s methodologies. This feature will most certainly be difficult to find in other shell prompts nowadays.
  \end{itemize}
  This has significant implications and contributes to the educational community.

  \framebreak

  For future work lines, some aspects may be considered:
  \begin{itemize}
    \item Guide the design towards Open/Close (OCP) principle.
    \item Provide fully structured usage help within the tool.
    \item Allow including extra information about students.
    \item Improve tool's current test structure.
    \item Add extra functionality to the commands that support the evaluation process, by including a set of libraries
    that perform specific tasks for this process.
  \end{itemize}
\end{frame}

%++++++++++++++++++++++++++++++++++++++++++++++++++++++++++++++++++++++++++++++ 

\section{Bibliografía}
\begin{frame}[allowframebreaks]
  \frametitle{Bibliografía}
  \bibliographystyle{ieeetr}
  \bibliography{presentacion_tfg}
  \nocite{*}
\end{frame}

\begin{frame}
  \frametitle{Fin de la presentación}
  \begin{center}
    \Huge{Gracias por su atención}
  \end{center}
\end{frame}

\end{document}
