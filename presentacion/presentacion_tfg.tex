\documentclass{beamer}
%\documentclass[xcolor=dvipsnames]{beamer}
\usepackage[spanish]{babel}
\usepackage[utf8]{inputenc}
\usepackage{graphicx}
\usepackage{latexsym}

\newcommand{\beamer}{\textsc{beamer}}
\newtheorem{definicion}{Definición}
\newtheorem{ejemplo}{Ejemplo}

%%%%%%%%%%%%%%%%%%%%%%%%%%%%%%%%%%%%%%%%%%%%%%%%%%%%%%%%%%%%%%%%%%%%%%%%%%%%%%%
\title[Trabajo de Fin de Grado]{ghedsh: Un intérprete de
comandos para GitHub Education\\
GitHub Education Shell: ghedsh.}

\author[Carlos de Armas Hernández] {
Autor: Carlos de Armas Hernández \\
Director: Casiano Rodríguez León
}

\institute[ULL]{Escuela Superior de Ingeniería y Tecnología \\
                Departamento de Ingeniería Informática y de Sistemas \\
                Universidad de La Laguna}
\date[12-07-2018]{12 de julio de 2018}
%%%%%%%%%%%%%%%%%%%%%%%%%%%%%%%%%%%%%%%%%%%%%%%%%%%%%%%%%%%%%%%%%%%%%%%%%%%%%%%

%\usetheme{Berlin}
\usetheme{Madrid}

%%%%%%%%%%%%%%%%%%%%%%%%%%%%%%%%%%%%%%%%%%%%%%%%%%%%%%%%%%%%%%%%%%%%%%%%%%%%%%%
\definecolor{pantone254}{RGB}{92,6,140}
\definecolor{pantone3015}{RGB}{92,6,140}
\definecolor{pantone432}{RGB}{92,6,140}
\setbeamercolor*{palette primary}{use=structure,fg=white,bg=pantone254}
\setbeamercolor*{palette secondary}{use=structure,fg=white,bg=pantone3015}
\setbeamercolor*{palette tertiary}{use=structure,fg=white,bg=pantone432}
\setbeamercolor*{palette sidebar primary}{use=structure,fg=pantone254}
\setbeamercolor*{palette sidebar tertiary}{use=structure,fg=pantone3015}
\setbeamercolor*{block title}{bg=pantone3015,fg=white}
\setbeamercolor*{alerted text}{fg=pantone432}
\setbeamercolor*{item projected}{fg=pantone254}
\setbeamercolor*{section in toc shaded}{use=structure,fg=structure.fg}
\setbeamercolor*{section in toc}{fg=pantone3015}
\setbeamercolor*{subsection in toc shaded}{fg=pantone3015}
\setbeamercolor*{subsection in toc}{fg=pantone432}

%%%%%%%%%%%%%%%%%%%%%%%%%%%%%%%%%%%%%%%%%%%%%%%%%%%%%%%%%%%%%%%%%%%%%%%%%%%%%%%
\begin{document}
  
%++++++++++++++++++++++++++++++++++++++++++++++++++++++++++++++++++++++++++++++  
\begin{frame}

  \includegraphics[width=0.3\textwidth]{img/ull.eps}
  \hspace*{7.5cm}
  \titlepage

\end{frame}
%++++++++++++++++++++++++++++++++++++++++++++++++++++++++++++++++++++++++++++++  

%++++++++++++++++++++++++++++++++++++++++++++++++++++++++++++++++++++++++++++++  
\begin{frame}
  \frametitle{Índice}  
  \tableofcontents
\end{frame}
%++++++++++++++++++++++++++++++++++++++++++++++++++++++++++++++++++++++++++++++  

\section{Introducción}
\begin{frame}[allowframebreaks,fragile]
  \frametitle{Introducción}
  \textbf{¿Qué es ``ghedsh''?}
  \bigskip

  Es una gema Ruby que consiste en un intérprete de comandos desarrollado para integrar las metodologías
  de GitHub Education, viendo las organizaciones como aulas y los repositorios como las asignaciones de los alumnos.
  
  \begin{center}
    \begin{figure}[!htb]  
      \minipage{0.5\textwidth}%
        \includegraphics[width=\linewidth]{img/ghedsh-logo.png}
      \endminipage
    \end{figure}
  \end{center}

  \framebreak
  %+++++++++++++++++++++++++++++++++++++++++++++++++++++++++++++++++++++++++++++++++++++++++++++++++++++++++++++++++++++++++++++++++++++++++++
  
  En cuanto a herramientas similares, existen las siguientes:
  \begin{itemize}
    \item {\it Teachers Pet}.
    \item {\it GitHub Classroom}.
    \item {\it ghi} (GitHub Issues).
    \item {\it ghs} (GitHub Search).
  \end{itemize}

  \framebreak
  Desarolladas por {\it GitHub}:
  \begin{itemize}
    \item \textbf{Teachers Pet}. {\it CLI} desarrollado previamente a {\it Classroom}.
    \begin{itemize}
      \item Inconveniente: los comandos se hacían excesivamente largos en determinados casos. 
      Cayó en desuso y se dejó de desarrollar.
    \end{itemize}
    \item \textbf{GitHub Classroom}. Una plataforma web que simplifica la configuración de las aulas y las asignaciones.
    Sin embargo, existen algunas limitaciones:
    \begin{itemize}
      \item Actualmente no dispone de alguna funcionalidad que permita al profesor crear un repositorio de evaluación.
      %para calificar las tareas.
      \item No da soporte a herramientas de integración continua ({\it Travis CI, CircleCI, Jenkins ...}).
      \item El sistema para añadir información adicional del alumno es incómodo de usar.
    \end{itemize}
  \end{itemize}

  \framebreak
  Desarrolladas por la comunidad:
  \begin{itemize}
    \item \textbf{ghi} (GitHub Issues). Permite gestionar las incidencias (issues) de los repositorios desde la terminal del usuario.
    \item \textbf{ghs} (GitHub Search). Permite realizar búsquedas de repositorios alojados en GitHub.
  \end{itemize}
\end{frame}
  %+++++++++++++++++++++++++++++++++++++++++++++++++++++++++++++++++++++++++++++++++++++++++++++++++++++++++++++++++++++++++++++++++++++++++++

\section{Objetivos}
\begin{frame}[fragile]
  \frametitle{Objetivos}
  
  Esta segunda versión de {\it ghedsh} busca mejorar el código fuente de la primera versión,
  teniendo en cuenta aspectos como la mantenibilidad del código y facilitar la incorporación de nuevas funcionalidades.
  \bigskip

  Por otro lado, una de las prioridades de esta herramienta es dar soporte al proceso de evaluación.
\end{frame}

%++++++++++++++++++++++++++++++++++++++++++++++++++++++++++++++++++++++++++++++  

\section{Tecnologías empleadas}
\begin{frame}[fragile]
  \frametitle{Tecnologías empleadas}
  \begin{columns}
    \column{0.3\linewidth}
       \centering
       \includegraphics[height=1.2cm, width=1.2cm]{img/ruby-logo.png}
     \column{0.6\linewidth}
        Lenguaje de programación: \textbf{Ruby}
  \end{columns}
  \bigskip

  \begin{columns}
    \column{0.3\linewidth}
       \centering
       \includegraphics[height=1cm, width=3.5cm]{img/bundler-logo.png}
     \column{0.6\linewidth}
        Gestión de dependencias: \textbf{Bundler}
  \end{columns}
  \bigskip

  \begin{columns}
    \column{0.3\linewidth}
       \centering
       \includegraphics[height=1.2cm, width=1.2cm]{img/rspec-logo.png}
     \column{0.6\linewidth}
        Testing: \textbf{RSpec}
  \end{columns}
\end{frame}
%++++++++++++++++++++++++++++++++++++++++++++++++++++++++++++++++++++++++++++++

\section{Desarrollo del proyecto}
\begin{frame}
\frametitle{Desarrollo del proyecto}
  Dividimos el desarrollo del proyecto en dos fases bien diferenciadas:

\end{frame}

\subsection{Primera fase. Análisis}
\begin{frame}[allowframebreaks,fragile]
\frametitle{Primera fase. Análisis}
\framesubtitle{Code Smell}
  
  XXXXXXXXXXXXXXXXXXXXXXXXXXXXXXXXXXXXXXXXXXXXXXXXX
  
\end{frame}
  %+++++++++++++++++++++++++++++++++++++++++++++++++++++++++++++++++++++++++++++++++++++++++++++++++++++++++
\subsection{Segunda fase. Refactorización}
\begin{frame}
\frametitle{Refactorización}

  XXXXXXXXXXXXXXXXXXXXXXXXXXXXXXXXXXXXXXXXXXX

\end{frame}

%+++++++++++++++++++++++++++++++++++++++++++++++++++++++++++++++++++++++++++++++++++++++++++++++++++++++++++


%+++++++++++++++++++++++++++++++++++++++++++++++++++++++++++++++++++++++++++++++++++++++++++++++++++++++++++

\section{Resultados obtenidos}

\begin{frame}[allowframebreaks]
\frametitle{Resultados obtenidos}


  \framebreak
  
  Las funcionalidades básicas son comunes a todos los roles que participan en la plataforma, 
  tanto alumnos como profesores, todos pueden hacer Log in, Log out y consultar un perfil.
  
  \bigskip
  
  Como alumno el usuario puede visitar el perfil, donde encontrará información básica de Github y dos pestañas, 
  en las que el alumno tiene un historial de las tareas que ha realizado de forma grupal e individual.
  
  \bigskip

  Los profesores tendrán el grupo de funcionalidades más completo, ya que ellos son los protagonistas 
  de la app.

  \framebreak

  Los profesores podrán desempeñar las siguientes tareas:

  \begin{itemize}
    \item Añadir una organización como aula.
    \item Invitar alumnos al aula.
    \item Crear una tarea.
    \item Añadir un fichero de alumnos asociado al aula.
    \item Editar las opciones del aula.
    \item Invitar alumnos a la tarea.
    \item Editar las opciones de la tarea.
    \item Crear un repositorio de evaluación de cada tarea.
  \end{itemize}

\end{frame}

%+++++++++++++++++++++++++++++++++++++++++++++++++++++++++++++++++++++++++++++++++++++++++++++++++++++++++++
  
\section{Caso de uso}

\begin{frame}[allowframebreaks]
\frametitle{Caso de uso}

  Con vistas a probar y testear que todo funcionaba de forma correcta, Casiano me sugirió probar 
  la plataforma para la realización de algunas prácticas individuales y grupales. Por ello, decidimos 
  realizar algunas tareas para la asignatura de Procesadores de Lenguajes en CodeLab.

  \framebreak

  XXXXXXXXXXXXXXXXXXXXXXXXXXX

\end{frame}

%+++++++++++++++++++++++++++++++++++++++++++++++++++++++++++++++++++++++++++++++++++++++++++++++++++++++++++

\section{Conclusions and future work lines}
\begin{frame}[allowframebreaks]
  \frametitle{Conclusions and future work lines}

  \begin{itemize}
    \item CodeLab was born as a tool that aims to extend the functionality of other tools such as Github 
          Classroom, adding specific functions for the teachers to support the management of courses and 
          the correction of programming labs.
    \item The platform has been designed with ease of use in mind for those who are not familiar with GitHub.
    \item Version control offers many advantages to developers. Most development companies use the git 
          version control system, and consequently it is essential that students learn to handle git correctly.
  \end{itemize}
  \framebreak
  %+++++++++++++++++++++++++++++++++++++++++++++++++++++++++++++++++++++++++++++++++++++++++++++++++++++++++++++++++++++++++++++++++++++++++++
  
  {\bf Future Work:}
  \begin{itemize}
    \item I would like to continue developing CodeLab, improving it and adding new functionalities. 
    \item One of the first improvements that is proposed is the use of a front-end library such as Vue 
          or React to improve the visual quality of the web platform.
    \item A new functionality that I would like to add is the possibility of more than one teacher per classroom.
  \end{itemize}
\end{frame}

%++++++++++++++++++++++++++++++++++++++++++++++++++++++++++++++++++++++++++++++ 

\section{Bibliografía}
\begin{frame}[allowframebreaks]
  \frametitle{Bibliografía}
  \bibliographystyle{ieeetr}
  \bibliography{presentacion_tfg}
  \nocite{*}
\end{frame}

\begin{frame}
  \frametitle{Fin de la presentación}
  \begin{center}
    \Huge{Gracias por su atención}
  \end{center}
\end{frame}

\end{document}
