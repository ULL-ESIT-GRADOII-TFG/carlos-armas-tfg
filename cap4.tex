%%%%%%%%%%%%%%%%%%%%%%%%%%%%%%%%%%%%%%%%%%%%%%%%%%%%%%%%%%%%%%%%%%%%%%%%%%%%%
% Chapter 4: Conclusiones y Trabajos Futuros 
%%%%%%%%%%%%%%%%%%%%%%%%%%%%%%%%%%%%%%%%%%%%%%%%%%%%%%%%%%%%%%%%%%%%%%%%%%%%%%%

%++++++++++++++++++++++++++++++++++++++++++++++++++++++++++++++++++++++++++++++

La segunda versión de la gema {\it ghedsh} ha alcanzado satisfactoriamente los objetivos planteados: mejorar su arquitectura inicial, con el fin de facilitar a otros desarrolladores
la incorporación de funcionalidades y, aparte, ser capaz de dar soporte al proceso de evaluación mediante las metodologías de {\it GitHub Education}, ofreciendo funcionalidades que difícilmente
podrán encontrarse actualmente en otros intérpretes de comandos similares. En este sentido, se ha realizado un aporte a la sociedad y, en concreto, a la comunidad educativa.
\bigskip

La retroalimentación es fundamental en el proceso de aprendizaje. Una comunicación eficaz con el alumnado puede marcar la diferencia. Mediante {\it ghedsh}, se puede acceder fácilmente a la interfaz que proporciona {\it GitHub} para comunicarse con los estudiantes, la cual permite comentar líneas y porciones de código en las asignaciones, logrando
mejorar cualitativamente las indicaciones del profesor. Ésto favorece que el estudiantado se realice preguntas acerca del código y comprender cómo es su funcionamiento.
\bigskip

Por otro lado, está previsto que el desarrollo de {\it ghedsh} tenga continuidad en el tiempo, puesto que, a pesar de que su diseño se ha optimizado notablemente, aún existe margen de mejora, lo que aumenta incluso más el potencial de esta herramienta.
\bigskip

Como desarrollo de líneas futuras, se pueden tener en cuenta diversos aspectos:
\begin{itemize}
	\item Orientar el diseño para cumplir totalmente el principio {\it Open/Closed} (OCP). Ésto lograría que {\it ghedsh} fuera extensible en funcionalidad mediante {\it plugins} desarrollados por otros programadores.
	\item Ofrecer una ayuda completa y estructurada dentro de la herramienta.
	\item Añadir información extendida de los alumnos, de manera que se pueda ver, por ejemplo, su perfil en el Campus Virtual.
	\item Enriquecer los comandos que dan soporte al proceso de evaluación, mediante un conjunto de librerías que realicen tareas especializadas para este proceso.
\end{itemize}



