Esta guía está especialmente dirigida a los desarrolladores que deseen ampliar o mejorar el diseño de {\it ghedsh}, así como incorporar nuevas funcionalidades.
\bigskip

Antes de empezar con el contenido, cabe comentar que se anima a que cada desarrollador incorpore nuevas ideas propias. Por lo tanto, cualquier
aspecto del diseño que sea mejorable, no dude en implementar su idea.

%---------------------------------------------------------------------------------
\section{Instalación}
\label{Apendice2:instalacion}

\subsection{Requisitos}
\label{subsec:b.1.1}

Node.js \textgreater = 8

\subsection{Dependencias}
\label{subsec:b.1.2}

Gitbook
Calibre

\subsection{Instalación}
\label{subsec:b.1.3}

Para instalar el paquete, basta con ejecutar el siguiente comando:
\begin{verbatim}
[~]$ npm install ghshell -g
\end{verbatim}

%---------------------------------------------------------------------------------
%---------------------------------------------------------------------------------

\section{Ejecución}
\label{Apendice2:ejecucion}

\begin{lstlisting}[language=JavaScript]
var foo = function(){
console.log('foo');
}
foo();
\end{lstlisting}
\bigskip

%---------------------------------------------------------------------------------
\subsection{Otras consideraciones}
\label{subsec:Apendice2.1}

Para que