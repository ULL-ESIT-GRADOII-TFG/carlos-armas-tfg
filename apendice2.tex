Esta guía está especialmente dirigida a los desarrolladores que deseen ampliar o mejorar el diseño de {\it ghedsh}, así como incorporar nuevas funcionalidades.
\bigskip

Antes de empezar con el contenido, cabe comentar que se anima a que cada desarrollador incorpore nuevas ideas propias. Por lo tanto, cualquier
aspecto del diseño que sea mejorable, no dude en implementar su idea.
\bigskip

%---------------------------------------------------------------------------------

\section{Requisitos}
\label{sec:1}

Versión mínima requerida para el desarrollo y uso: \verb 2.4.0  . 

\section{Estructura de la gema}
\label{sec:2}
En esta sección se explicará con un poco más de detalle (respecto a lo ya explicado en las secciones \ref{2.1.1} y \ref{2.1.2}) los ficheros más destacables de la gema.
\bigskip

En cuanto a la raíz:
\begin{itemize}
  \item \verb Rakefile : Automatiza ejecución de tareas. Actualmente, están definidas \textbf{test} (RSpec) y \textbf{publish}. \textbf{Test} ejecuta los tests definidos y \textbf{publish} automatiza el proceso de publicar la gema en Rubygems.
  \item \verb ghedsh.gemspec : contiene información de la gema {\it ghedsh}. Aquí debemos añadir las dependencias, especificando una versión o rango de versiones permitidas. También comentar que, si en {\it ghedsh} se hace uso de alguna funcionalidad específica de una versión de Ruby, debemos especificar la versión de Ruby requerida al usuario.
  Ésto es importante, puesto que evitará quebraderos de cabeza pensando que el error se encuentra en nuestro código. En {\it ghedsh} tenemos \verb s.required_ruby_version  \verb =  \verb ">= 2.4.0"  .
\end{itemize}
\bigskip

Directorio \verb /spec :
\begin{itemize}
  \item \verb spec_helper.rb : contiene la configuración de RSpec.
  \item \verb cli_spec.rb : contiene la estructura básica de los tests.
\end{itemize}
\bigskip

Directorio \verb /file_templates :
\begin{itemize}
  \item \verb add_members_template.json : estructura del fichero para añadir miembros a una organización.
  \item \verb create_teams_template.json : estructura del fichero para la creación masiva de equipos y sus miembros.
  \item \verb invite_outside_collabs.json : estructura del fichero para invitar a ser miembros de la organización los que son colaboradores externos.
  \item \verb remove_members_template.json : estructura del fichero para eliminar masivamente miembros de la organización.

\end{itemize}
\bigskip

Puesto que el contenido de \verb /lib  es el código fuente de la gema y es el más importante, lo comentaremos aparte en la siguiente sección (\ref{sec:2.1}).

\subsection{Directorio lib}
\label{sec:2.1}
\bigskip

\subsubsection{version.rb}
Indica la versión actual de la gema usando {\it Semantic Versioning}. Conviene ir actualizándola según vayamos añadiendo funcionalidades o arreglando errores. Sería buena idea incluso realizar una pequeña descripción
de los nuevos cambios en cada versión.

\subsubsection{interface.rb}

Como ya se ha explicado en anteriores secciones (\ref{2.1.2}), la clase \verb Interface  se encarga del REPL.
\bigskip

Veamos desglosado en pasos el flujo de ejecución de {\it ghedsh}:
\begin{itemize}
  \item Desde \verb /bin  se instancia \verb Interface , y se llama al método \verb start .
  \item \verb start  primero comprueba si se ha especificado algún parámetro al iniciar {\it ghedsh} (por ejemplo, \verb ghedsh  \verb --version ). A continuación, se inicializa el contexto (clase \verb ShellContext ) y de esta manera conocer la configuración de la última sesión del usuario en {\it ghedsh} y los comandos que soporta esta herramienta.
  \item Inicio del REPL.
\end{itemize}

\subsubsection{helpers.rb}

En este fichero se encuentran funciones que son utilizadas frecuentemente por las acciones (\verb /actions ). Entre ellas se encuentran, por ejemplo, el menú para la creación de repositorios, construir una expresión regular a partir de una cadena ({\it String}), etcétera.

\subsubsection{context.rb}

Contiene la clase que representa el contexto. Más adelante se explicará en detalle el significado de este concepto, fundamental en {\it ghedsh}.
Accede al directorio de configuración y restaura el estado de la última sesión del usuario.
\bigskip

Las variables más importantes de esta clase son:
\begin{itemize}
  \item \verb @deep : referencia a la clase (\verb User , \verb Organization , \verb Team ) en la que se encuentra el usuario y permite realizar las acciones de cada una de ellas.
  \item \verb @config : es el {\it Hash} que, dependiendo de su contenido, \verb @deep  referenciará a una clase u otra. (Véase su estructura interna en \ref{3.2.2}).
\end{itemize}

\subsubsection{commands.rb}

Clase que contiene los comandos. Los comandos son funciones que se exportan bajo un nombre (\verb orgs , \verb repos , \verb files  ...). Internamente, comprueba si la clase a la que referencia \verb @deep , implementa la acción solicitada.
En caso afirmativo, se ejecuta la acción. En caso negativo, se informa al usuario que el comando no se puede llevar a cabo en el contexto actual.

\section{Concepto de contexto}
\label{apen:3}


Es uno de los términos fundamentales en {\it ghedsh} (ver también \ref{3.2.2}). En esencia, el contexto es una estructura de datos (en este caso, {\it hash} de Ruby). Veamos varios ejemplos para saber interpretarlo:
\bigskip
\bigskip
\bigskip 

\begin{minipage}[t]{0.5\linewidth}
  1.a) :
  \begin{lstlisting}[language=Ruby]
    config = {
    'User' => client.login.to_s,
  	'user_url' => "https://github.com/your-profile",
    'Org' => nil,
    'org_url' => nil,
    'Repo' => nil,
    'repo_url' => nil,
    'Team' => nil,
    'team_url' => nil
  }
  \end{lstlisting}
\end{minipage}
  %
\begin{minipage}[t]{0.5\linewidth}
    1.b) :
  \begin{lstlisting}[language=Ruby]
    config = {
    'User' => client.login.to_s,
  	'user_url' => "https://github.com/your-profile",
    'Org' => nil,
    'org_url' => nil,
    'Repo' => "my-user-repo",
    'repo_url' => nil,
    'Team' => nil,
    'team_url' => nil
  }
  \end{lstlisting}
\end{minipage}

\begin{itemize}
  \item 1.a: Contexto de usuario, entonces \verb @deep  \verb =  \verb User  .
  \item 1.b: Contexto repositorio de usuario ({\it my-user-repo}), se mantiene \verb @deep  \verb =  \verb User  .
\end{itemize}
\bigskip

\begin{minipage}[t]{0.5\linewidth}
  2.a) :
  \begin{lstlisting}[language=Ruby]
    config = {
    'User' => client.login.to_s,
  	'user_url' => "https://github.com/your-profile",
    'Org' => "MY-ORG",
    'org_url' => nil,
    'Repo' => nil,
    'repo_url' => nil,
    'Team' => nil,
    'team_url' => nil
  }
  \end{lstlisting}
\end{minipage}
  %
\begin{minipage}[t]{0.5\linewidth}
    2.b) :
  \begin{lstlisting}[language=Ruby]
    config = {
    'User' => client.login.to_s,
  	'user_url' => "https://github.com/your-profile",
    'Org' => "MY-ORG",
    'org_url' => nil,
    'Repo' => "MY-ORG-repo",
    'repo_url' => nil,
    'Team' => nil,
    'team_url' => nil
  }
  \end{lstlisting}
\end{minipage}

\begin{itemize}
  \item 2.a: Contexto de organización ({\it MY-ORG}), por lo tanto \verb @deep  \verb =  \verb Organization . 
  \item 2.b: Contexto repositorio de organización ({\it MY-ORG-repo}), se mantiene \verb @deep  \verb =  \verb Organization  .
\end{itemize}
\bigskip

\begin{minipage}[t]{0.8\linewidth}
  3) :
  \begin{lstlisting}[language=Ruby]
    config = {
    'User' => client.login.to_s,
  	'user_url' => "https://github.com/your-profile",
    'Org' => "MY-ORG",
    'org_url' => nil,
    'Repo' => nil,
    'repo_url' => nil,
    'Team' => "MY-ORG-team",
    'team_url' => nil
  }
  \end{lstlisting}
\end{minipage}

\begin{itemize}
  \item 3 : Contexto de equipo ({\it MY-ORG-team}) dentro de la organización {\it MY-ORG}, ahora \verb @deep  \verb =  \verb Team .
\end{itemize}


\section{Cómo añadir nuevos comandos}
\label{apen:4}

En esta última sección veremos cómo añadir comandos. Hay que tener en cuenta que uno de los objetivos deseables para desarrollos futuros es cumplir totalmente el principio {\it Open/Closed}. Por lo tanto,
habría que cambiar la forma de añadirlos. No obstante, conviene conocer cómo se realiza esta tarea en la segunda versión de {\it ghedsh} para ayudar a generar nuevas ideas.
\bigskip

Veámoslo con un ejemplo. Supongamos que queremos añadir un nuevo comando, cuyo objetivo es mostrar los nombres de las plantillas \verb .gitignore  disponibles en {\it GitHub}.
\bigskip

Primero, vamos a decidir cómo llamaremos el comando, por ejemplo, \verb templates  y vamos a pensar para qué contexto estaría disponible (consideramos en este caso sólo contexto usuario). Lo siguiente que tenemos que hacer es ir al fichero \verb commands.rb  y definir una función y exportarla con \verb add_command , suponemos que la llamamos \verb show_templates .
Quedaría lo siguiente:
\bigskip

\verb lib/commmands.rb :
\begin{lstlisting}[language=Ruby]
  # params seria algun parametro proporcionado por el usuario
  # por ejemplo, dada una inicial, se mostraran las plantillas que empiecen por ella
  def show_templates(params)
    if @enviroment.deep.method_defined? :create_issue
      @enviroment.deep.new.templates(@enviroment.client, params)
    else
      puts Rainbow("Command not available in context \"#{@enviroment.deep.name}\"").color(WARNING_CODE)
    end
  end
}
\end{lstlisting}
\bigskip

Ahora definimos la acción en la clase \verb User . Si la queremos en más contextos, definimos la acción en particular en la correspondiente clase.
\bigskip

\verb lib/actions/user.rb :
\begin{lstlisting}[language=Ruby]
  # client es el objeto Octokit de la GitHub API
  # Documentacion en http://octokit.github.io/octokit.rb/Octokit/Client.html
  def templates(client, params)
    letter = params[0]
    client.gitignore_templates.each do |name|
      puts name if name.start_with?(letter)
    end
  end
}
\end{lstlisting}
\bigskip

El usuario emplearía el comando de la siguiente manera: \verb templates  \verb N  .

El resultado sería:
\begin{verbatim}
  Nim
\end{verbatim}

\begin{verbatim}
  Node
\end{verbatim}

